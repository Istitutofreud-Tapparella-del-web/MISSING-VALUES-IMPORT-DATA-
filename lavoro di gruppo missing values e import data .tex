\documentclass[11pt]{article}

    \usepackage[breakable]{tcolorbox}
    \usepackage{parskip} % Stop auto-indenting (to mimic markdown behaviour)
    

    % Basic figure setup, for now with no caption control since it's done
    % automatically by Pandoc (which extracts ![](path) syntax from Markdown).
    \usepackage{graphicx}
    % Maintain compatibility with old templates. Remove in nbconvert 6.0
    \let\Oldincludegraphics\includegraphics
    % Ensure that by default, figures have no caption (until we provide a
    % proper Figure object with a Caption API and a way to capture that
    % in the conversion process - todo).
    \usepackage{caption}
    \DeclareCaptionFormat{nocaption}{}
    \captionsetup{format=nocaption,aboveskip=0pt,belowskip=0pt}

    \usepackage{float}
    \floatplacement{figure}{H} % forces figures to be placed at the correct location
    \usepackage{xcolor} % Allow colors to be defined
    \usepackage{enumerate} % Needed for markdown enumerations to work
    \usepackage{geometry} % Used to adjust the document margins
    \usepackage{amsmath} % Equations
    \usepackage{amssymb} % Equations
    \usepackage{textcomp} % defines textquotesingle
    % Hack from http://tex.stackexchange.com/a/47451/13684:
    \AtBeginDocument{%
        \def\PYZsq{\textquotesingle}% Upright quotes in Pygmentized code
    }
    \usepackage{upquote} % Upright quotes for verbatim code
    \usepackage{eurosym} % defines \euro

    \usepackage{iftex}
    \ifPDFTeX
        \usepackage[T1]{fontenc}
        \IfFileExists{alphabeta.sty}{
              \usepackage{alphabeta}
          }{
              \usepackage[mathletters]{ucs}
              \usepackage[utf8x]{inputenc}
          }
    \else
        \usepackage{fontspec}
        \usepackage{unicode-math}
    \fi

    \usepackage{fancyvrb} % verbatim replacement that allows latex
    \usepackage{grffile} % extends the file name processing of package graphics
                         % to support a larger range
    \makeatletter % fix for old versions of grffile with XeLaTeX
    \@ifpackagelater{grffile}{2019/11/01}
    {
      % Do nothing on new versions
    }
    {
      \def\Gread@@xetex#1{%
        \IfFileExists{"\Gin@base".bb}%
        {\Gread@eps{\Gin@base.bb}}%
        {\Gread@@xetex@aux#1}%
      }
    }
    \makeatother
    \usepackage[Export]{adjustbox} % Used to constrain images to a maximum size
    \adjustboxset{max size={0.9\linewidth}{0.9\paperheight}}

    % The hyperref package gives us a pdf with properly built
    % internal navigation ('pdf bookmarks' for the table of contents,
    % internal cross-reference links, web links for URLs, etc.)
    \usepackage{hyperref}
    % The default LaTeX title has an obnoxious amount of whitespace. By default,
    % titling removes some of it. It also provides customization options.
    \usepackage{titling}
    \usepackage{longtable} % longtable support required by pandoc >1.10
    \usepackage{booktabs}  % table support for pandoc > 1.12.2
    \usepackage{array}     % table support for pandoc >= 2.11.3
    \usepackage{calc}      % table minipage width calculation for pandoc >= 2.11.1
    \usepackage[inline]{enumitem} % IRkernel/repr support (it uses the enumerate* environment)
    \usepackage[normalem]{ulem} % ulem is needed to support strikethroughs (\sout)
                                % normalem makes italics be italics, not underlines
    \usepackage{mathrsfs}
    

    
    % Colors for the hyperref package
    \definecolor{urlcolor}{rgb}{0,.145,.698}
    \definecolor{linkcolor}{rgb}{.71,0.21,0.01}
    \definecolor{citecolor}{rgb}{.12,.54,.11}

    % ANSI colors
    \definecolor{ansi-black}{HTML}{3E424D}
    \definecolor{ansi-black-intense}{HTML}{282C36}
    \definecolor{ansi-red}{HTML}{E75C58}
    \definecolor{ansi-red-intense}{HTML}{B22B31}
    \definecolor{ansi-green}{HTML}{00A250}
    \definecolor{ansi-green-intense}{HTML}{007427}
    \definecolor{ansi-yellow}{HTML}{DDB62B}
    \definecolor{ansi-yellow-intense}{HTML}{B27D12}
    \definecolor{ansi-blue}{HTML}{208FFB}
    \definecolor{ansi-blue-intense}{HTML}{0065CA}
    \definecolor{ansi-magenta}{HTML}{D160C4}
    \definecolor{ansi-magenta-intense}{HTML}{A03196}
    \definecolor{ansi-cyan}{HTML}{60C6C8}
    \definecolor{ansi-cyan-intense}{HTML}{258F8F}
    \definecolor{ansi-white}{HTML}{C5C1B4}
    \definecolor{ansi-white-intense}{HTML}{A1A6B2}
    \definecolor{ansi-default-inverse-fg}{HTML}{FFFFFF}
    \definecolor{ansi-default-inverse-bg}{HTML}{000000}

    % common color for the border for error outputs.
    \definecolor{outerrorbackground}{HTML}{FFDFDF}

    % commands and environments needed by pandoc snippets
    % extracted from the output of `pandoc -s`
    \providecommand{\tightlist}{%
      \setlength{\itemsep}{0pt}\setlength{\parskip}{0pt}}
    \DefineVerbatimEnvironment{Highlighting}{Verbatim}{commandchars=\\\{\}}
    % Add ',fontsize=\small' for more characters per line
    \newenvironment{Shaded}{}{}
    \newcommand{\KeywordTok}[1]{\textcolor[rgb]{0.00,0.44,0.13}{\textbf{{#1}}}}
    \newcommand{\DataTypeTok}[1]{\textcolor[rgb]{0.56,0.13,0.00}{{#1}}}
    \newcommand{\DecValTok}[1]{\textcolor[rgb]{0.25,0.63,0.44}{{#1}}}
    \newcommand{\BaseNTok}[1]{\textcolor[rgb]{0.25,0.63,0.44}{{#1}}}
    \newcommand{\FloatTok}[1]{\textcolor[rgb]{0.25,0.63,0.44}{{#1}}}
    \newcommand{\CharTok}[1]{\textcolor[rgb]{0.25,0.44,0.63}{{#1}}}
    \newcommand{\StringTok}[1]{\textcolor[rgb]{0.25,0.44,0.63}{{#1}}}
    \newcommand{\CommentTok}[1]{\textcolor[rgb]{0.38,0.63,0.69}{\textit{{#1}}}}
    \newcommand{\OtherTok}[1]{\textcolor[rgb]{0.00,0.44,0.13}{{#1}}}
    \newcommand{\AlertTok}[1]{\textcolor[rgb]{1.00,0.00,0.00}{\textbf{{#1}}}}
    \newcommand{\FunctionTok}[1]{\textcolor[rgb]{0.02,0.16,0.49}{{#1}}}
    \newcommand{\RegionMarkerTok}[1]{{#1}}
    \newcommand{\ErrorTok}[1]{\textcolor[rgb]{1.00,0.00,0.00}{\textbf{{#1}}}}
    \newcommand{\NormalTok}[1]{{#1}}

    % Additional commands for more recent versions of Pandoc
    \newcommand{\ConstantTok}[1]{\textcolor[rgb]{0.53,0.00,0.00}{{#1}}}
    \newcommand{\SpecialCharTok}[1]{\textcolor[rgb]{0.25,0.44,0.63}{{#1}}}
    \newcommand{\VerbatimStringTok}[1]{\textcolor[rgb]{0.25,0.44,0.63}{{#1}}}
    \newcommand{\SpecialStringTok}[1]{\textcolor[rgb]{0.73,0.40,0.53}{{#1}}}
    \newcommand{\ImportTok}[1]{{#1}}
    \newcommand{\DocumentationTok}[1]{\textcolor[rgb]{0.73,0.13,0.13}{\textit{{#1}}}}
    \newcommand{\AnnotationTok}[1]{\textcolor[rgb]{0.38,0.63,0.69}{\textbf{\textit{{#1}}}}}
    \newcommand{\CommentVarTok}[1]{\textcolor[rgb]{0.38,0.63,0.69}{\textbf{\textit{{#1}}}}}
    \newcommand{\VariableTok}[1]{\textcolor[rgb]{0.10,0.09,0.49}{{#1}}}
    \newcommand{\ControlFlowTok}[1]{\textcolor[rgb]{0.00,0.44,0.13}{\textbf{{#1}}}}
    \newcommand{\OperatorTok}[1]{\textcolor[rgb]{0.40,0.40,0.40}{{#1}}}
    \newcommand{\BuiltInTok}[1]{{#1}}
    \newcommand{\ExtensionTok}[1]{{#1}}
    \newcommand{\PreprocessorTok}[1]{\textcolor[rgb]{0.74,0.48,0.00}{{#1}}}
    \newcommand{\AttributeTok}[1]{\textcolor[rgb]{0.49,0.56,0.16}{{#1}}}
    \newcommand{\InformationTok}[1]{\textcolor[rgb]{0.38,0.63,0.69}{\textbf{\textit{{#1}}}}}
    \newcommand{\WarningTok}[1]{\textcolor[rgb]{0.38,0.63,0.69}{\textbf{\textit{{#1}}}}}


    % Define a nice break command that doesn't care if a line doesn't already
    % exist.
    \def\br{\hspace*{\fill} \\* }
    % Math Jax compatibility definitions
    \def\gt{>}
    \def\lt{<}
    \let\Oldtex\TeX
    \let\Oldlatex\LaTeX
    \renewcommand{\TeX}{\textrm{\Oldtex}}
    \renewcommand{\LaTeX}{\textrm{\Oldlatex}}
    % Document parameters
    % Document title
    \title{lavoro di gruppo missing values e import data }
    
    
    
    
    
% Pygments definitions
\makeatletter
\def\PY@reset{\let\PY@it=\relax \let\PY@bf=\relax%
    \let\PY@ul=\relax \let\PY@tc=\relax%
    \let\PY@bc=\relax \let\PY@ff=\relax}
\def\PY@tok#1{\csname PY@tok@#1\endcsname}
\def\PY@toks#1+{\ifx\relax#1\empty\else%
    \PY@tok{#1}\expandafter\PY@toks\fi}
\def\PY@do#1{\PY@bc{\PY@tc{\PY@ul{%
    \PY@it{\PY@bf{\PY@ff{#1}}}}}}}
\def\PY#1#2{\PY@reset\PY@toks#1+\relax+\PY@do{#2}}

\@namedef{PY@tok@w}{\def\PY@tc##1{\textcolor[rgb]{0.73,0.73,0.73}{##1}}}
\@namedef{PY@tok@c}{\let\PY@it=\textit\def\PY@tc##1{\textcolor[rgb]{0.24,0.48,0.48}{##1}}}
\@namedef{PY@tok@cp}{\def\PY@tc##1{\textcolor[rgb]{0.61,0.40,0.00}{##1}}}
\@namedef{PY@tok@k}{\let\PY@bf=\textbf\def\PY@tc##1{\textcolor[rgb]{0.00,0.50,0.00}{##1}}}
\@namedef{PY@tok@kp}{\def\PY@tc##1{\textcolor[rgb]{0.00,0.50,0.00}{##1}}}
\@namedef{PY@tok@kt}{\def\PY@tc##1{\textcolor[rgb]{0.69,0.00,0.25}{##1}}}
\@namedef{PY@tok@o}{\def\PY@tc##1{\textcolor[rgb]{0.40,0.40,0.40}{##1}}}
\@namedef{PY@tok@ow}{\let\PY@bf=\textbf\def\PY@tc##1{\textcolor[rgb]{0.67,0.13,1.00}{##1}}}
\@namedef{PY@tok@nb}{\def\PY@tc##1{\textcolor[rgb]{0.00,0.50,0.00}{##1}}}
\@namedef{PY@tok@nf}{\def\PY@tc##1{\textcolor[rgb]{0.00,0.00,1.00}{##1}}}
\@namedef{PY@tok@nc}{\let\PY@bf=\textbf\def\PY@tc##1{\textcolor[rgb]{0.00,0.00,1.00}{##1}}}
\@namedef{PY@tok@nn}{\let\PY@bf=\textbf\def\PY@tc##1{\textcolor[rgb]{0.00,0.00,1.00}{##1}}}
\@namedef{PY@tok@ne}{\let\PY@bf=\textbf\def\PY@tc##1{\textcolor[rgb]{0.80,0.25,0.22}{##1}}}
\@namedef{PY@tok@nv}{\def\PY@tc##1{\textcolor[rgb]{0.10,0.09,0.49}{##1}}}
\@namedef{PY@tok@no}{\def\PY@tc##1{\textcolor[rgb]{0.53,0.00,0.00}{##1}}}
\@namedef{PY@tok@nl}{\def\PY@tc##1{\textcolor[rgb]{0.46,0.46,0.00}{##1}}}
\@namedef{PY@tok@ni}{\let\PY@bf=\textbf\def\PY@tc##1{\textcolor[rgb]{0.44,0.44,0.44}{##1}}}
\@namedef{PY@tok@na}{\def\PY@tc##1{\textcolor[rgb]{0.41,0.47,0.13}{##1}}}
\@namedef{PY@tok@nt}{\let\PY@bf=\textbf\def\PY@tc##1{\textcolor[rgb]{0.00,0.50,0.00}{##1}}}
\@namedef{PY@tok@nd}{\def\PY@tc##1{\textcolor[rgb]{0.67,0.13,1.00}{##1}}}
\@namedef{PY@tok@s}{\def\PY@tc##1{\textcolor[rgb]{0.73,0.13,0.13}{##1}}}
\@namedef{PY@tok@sd}{\let\PY@it=\textit\def\PY@tc##1{\textcolor[rgb]{0.73,0.13,0.13}{##1}}}
\@namedef{PY@tok@si}{\let\PY@bf=\textbf\def\PY@tc##1{\textcolor[rgb]{0.64,0.35,0.47}{##1}}}
\@namedef{PY@tok@se}{\let\PY@bf=\textbf\def\PY@tc##1{\textcolor[rgb]{0.67,0.36,0.12}{##1}}}
\@namedef{PY@tok@sr}{\def\PY@tc##1{\textcolor[rgb]{0.64,0.35,0.47}{##1}}}
\@namedef{PY@tok@ss}{\def\PY@tc##1{\textcolor[rgb]{0.10,0.09,0.49}{##1}}}
\@namedef{PY@tok@sx}{\def\PY@tc##1{\textcolor[rgb]{0.00,0.50,0.00}{##1}}}
\@namedef{PY@tok@m}{\def\PY@tc##1{\textcolor[rgb]{0.40,0.40,0.40}{##1}}}
\@namedef{PY@tok@gh}{\let\PY@bf=\textbf\def\PY@tc##1{\textcolor[rgb]{0.00,0.00,0.50}{##1}}}
\@namedef{PY@tok@gu}{\let\PY@bf=\textbf\def\PY@tc##1{\textcolor[rgb]{0.50,0.00,0.50}{##1}}}
\@namedef{PY@tok@gd}{\def\PY@tc##1{\textcolor[rgb]{0.63,0.00,0.00}{##1}}}
\@namedef{PY@tok@gi}{\def\PY@tc##1{\textcolor[rgb]{0.00,0.52,0.00}{##1}}}
\@namedef{PY@tok@gr}{\def\PY@tc##1{\textcolor[rgb]{0.89,0.00,0.00}{##1}}}
\@namedef{PY@tok@ge}{\let\PY@it=\textit}
\@namedef{PY@tok@gs}{\let\PY@bf=\textbf}
\@namedef{PY@tok@gp}{\let\PY@bf=\textbf\def\PY@tc##1{\textcolor[rgb]{0.00,0.00,0.50}{##1}}}
\@namedef{PY@tok@go}{\def\PY@tc##1{\textcolor[rgb]{0.44,0.44,0.44}{##1}}}
\@namedef{PY@tok@gt}{\def\PY@tc##1{\textcolor[rgb]{0.00,0.27,0.87}{##1}}}
\@namedef{PY@tok@err}{\def\PY@bc##1{{\setlength{\fboxsep}{\string -\fboxrule}\fcolorbox[rgb]{1.00,0.00,0.00}{1,1,1}{\strut ##1}}}}
\@namedef{PY@tok@kc}{\let\PY@bf=\textbf\def\PY@tc##1{\textcolor[rgb]{0.00,0.50,0.00}{##1}}}
\@namedef{PY@tok@kd}{\let\PY@bf=\textbf\def\PY@tc##1{\textcolor[rgb]{0.00,0.50,0.00}{##1}}}
\@namedef{PY@tok@kn}{\let\PY@bf=\textbf\def\PY@tc##1{\textcolor[rgb]{0.00,0.50,0.00}{##1}}}
\@namedef{PY@tok@kr}{\let\PY@bf=\textbf\def\PY@tc##1{\textcolor[rgb]{0.00,0.50,0.00}{##1}}}
\@namedef{PY@tok@bp}{\def\PY@tc##1{\textcolor[rgb]{0.00,0.50,0.00}{##1}}}
\@namedef{PY@tok@fm}{\def\PY@tc##1{\textcolor[rgb]{0.00,0.00,1.00}{##1}}}
\@namedef{PY@tok@vc}{\def\PY@tc##1{\textcolor[rgb]{0.10,0.09,0.49}{##1}}}
\@namedef{PY@tok@vg}{\def\PY@tc##1{\textcolor[rgb]{0.10,0.09,0.49}{##1}}}
\@namedef{PY@tok@vi}{\def\PY@tc##1{\textcolor[rgb]{0.10,0.09,0.49}{##1}}}
\@namedef{PY@tok@vm}{\def\PY@tc##1{\textcolor[rgb]{0.10,0.09,0.49}{##1}}}
\@namedef{PY@tok@sa}{\def\PY@tc##1{\textcolor[rgb]{0.73,0.13,0.13}{##1}}}
\@namedef{PY@tok@sb}{\def\PY@tc##1{\textcolor[rgb]{0.73,0.13,0.13}{##1}}}
\@namedef{PY@tok@sc}{\def\PY@tc##1{\textcolor[rgb]{0.73,0.13,0.13}{##1}}}
\@namedef{PY@tok@dl}{\def\PY@tc##1{\textcolor[rgb]{0.73,0.13,0.13}{##1}}}
\@namedef{PY@tok@s2}{\def\PY@tc##1{\textcolor[rgb]{0.73,0.13,0.13}{##1}}}
\@namedef{PY@tok@sh}{\def\PY@tc##1{\textcolor[rgb]{0.73,0.13,0.13}{##1}}}
\@namedef{PY@tok@s1}{\def\PY@tc##1{\textcolor[rgb]{0.73,0.13,0.13}{##1}}}
\@namedef{PY@tok@mb}{\def\PY@tc##1{\textcolor[rgb]{0.40,0.40,0.40}{##1}}}
\@namedef{PY@tok@mf}{\def\PY@tc##1{\textcolor[rgb]{0.40,0.40,0.40}{##1}}}
\@namedef{PY@tok@mh}{\def\PY@tc##1{\textcolor[rgb]{0.40,0.40,0.40}{##1}}}
\@namedef{PY@tok@mi}{\def\PY@tc##1{\textcolor[rgb]{0.40,0.40,0.40}{##1}}}
\@namedef{PY@tok@il}{\def\PY@tc##1{\textcolor[rgb]{0.40,0.40,0.40}{##1}}}
\@namedef{PY@tok@mo}{\def\PY@tc##1{\textcolor[rgb]{0.40,0.40,0.40}{##1}}}
\@namedef{PY@tok@ch}{\let\PY@it=\textit\def\PY@tc##1{\textcolor[rgb]{0.24,0.48,0.48}{##1}}}
\@namedef{PY@tok@cm}{\let\PY@it=\textit\def\PY@tc##1{\textcolor[rgb]{0.24,0.48,0.48}{##1}}}
\@namedef{PY@tok@cpf}{\let\PY@it=\textit\def\PY@tc##1{\textcolor[rgb]{0.24,0.48,0.48}{##1}}}
\@namedef{PY@tok@c1}{\let\PY@it=\textit\def\PY@tc##1{\textcolor[rgb]{0.24,0.48,0.48}{##1}}}
\@namedef{PY@tok@cs}{\let\PY@it=\textit\def\PY@tc##1{\textcolor[rgb]{0.24,0.48,0.48}{##1}}}

\def\PYZbs{\char`\\}
\def\PYZus{\char`\_}
\def\PYZob{\char`\{}
\def\PYZcb{\char`\}}
\def\PYZca{\char`\^}
\def\PYZam{\char`\&}
\def\PYZlt{\char`\<}
\def\PYZgt{\char`\>}
\def\PYZsh{\char`\#}
\def\PYZpc{\char`\%}
\def\PYZdl{\char`\$}
\def\PYZhy{\char`\-}
\def\PYZsq{\char`\'}
\def\PYZdq{\char`\"}
\def\PYZti{\char`\~}
% for compatibility with earlier versions
\def\PYZat{@}
\def\PYZlb{[}
\def\PYZrb{]}
\makeatother


    % For linebreaks inside Verbatim environment from package fancyvrb.
    \makeatletter
        \newbox\Wrappedcontinuationbox
        \newbox\Wrappedvisiblespacebox
        \newcommand*\Wrappedvisiblespace {\textcolor{red}{\textvisiblespace}}
        \newcommand*\Wrappedcontinuationsymbol {\textcolor{red}{\llap{\tiny$\m@th\hookrightarrow$}}}
        \newcommand*\Wrappedcontinuationindent {3ex }
        \newcommand*\Wrappedafterbreak {\kern\Wrappedcontinuationindent\copy\Wrappedcontinuationbox}
        % Take advantage of the already applied Pygments mark-up to insert
        % potential linebreaks for TeX processing.
        %        {, <, #, %, $, ' and ": go to next line.
        %        _, }, ^, &, >, - and ~: stay at end of broken line.
        % Use of \textquotesingle for straight quote.
        \newcommand*\Wrappedbreaksatspecials {%
            \def\PYGZus{\discretionary{\char`\_}{\Wrappedafterbreak}{\char`\_}}%
            \def\PYGZob{\discretionary{}{\Wrappedafterbreak\char`\{}{\char`\{}}%
            \def\PYGZcb{\discretionary{\char`\}}{\Wrappedafterbreak}{\char`\}}}%
            \def\PYGZca{\discretionary{\char`\^}{\Wrappedafterbreak}{\char`\^}}%
            \def\PYGZam{\discretionary{\char`\&}{\Wrappedafterbreak}{\char`\&}}%
            \def\PYGZlt{\discretionary{}{\Wrappedafterbreak\char`\<}{\char`\<}}%
            \def\PYGZgt{\discretionary{\char`\>}{\Wrappedafterbreak}{\char`\>}}%
            \def\PYGZsh{\discretionary{}{\Wrappedafterbreak\char`\#}{\char`\#}}%
            \def\PYGZpc{\discretionary{}{\Wrappedafterbreak\char`\%}{\char`\%}}%
            \def\PYGZdl{\discretionary{}{\Wrappedafterbreak\char`\$}{\char`\$}}%
            \def\PYGZhy{\discretionary{\char`\-}{\Wrappedafterbreak}{\char`\-}}%
            \def\PYGZsq{\discretionary{}{\Wrappedafterbreak\textquotesingle}{\textquotesingle}}%
            \def\PYGZdq{\discretionary{}{\Wrappedafterbreak\char`\"}{\char`\"}}%
            \def\PYGZti{\discretionary{\char`\~}{\Wrappedafterbreak}{\char`\~}}%
        }
        % Some characters . , ; ? ! / are not pygmentized.
        % This macro makes them "active" and they will insert potential linebreaks
        \newcommand*\Wrappedbreaksatpunct {%
            \lccode`\~`\.\lowercase{\def~}{\discretionary{\hbox{\char`\.}}{\Wrappedafterbreak}{\hbox{\char`\.}}}%
            \lccode`\~`\,\lowercase{\def~}{\discretionary{\hbox{\char`\,}}{\Wrappedafterbreak}{\hbox{\char`\,}}}%
            \lccode`\~`\;\lowercase{\def~}{\discretionary{\hbox{\char`\;}}{\Wrappedafterbreak}{\hbox{\char`\;}}}%
            \lccode`\~`\:\lowercase{\def~}{\discretionary{\hbox{\char`\:}}{\Wrappedafterbreak}{\hbox{\char`\:}}}%
            \lccode`\~`\?\lowercase{\def~}{\discretionary{\hbox{\char`\?}}{\Wrappedafterbreak}{\hbox{\char`\?}}}%
            \lccode`\~`\!\lowercase{\def~}{\discretionary{\hbox{\char`\!}}{\Wrappedafterbreak}{\hbox{\char`\!}}}%
            \lccode`\~`\/\lowercase{\def~}{\discretionary{\hbox{\char`\/}}{\Wrappedafterbreak}{\hbox{\char`\/}}}%
            \catcode`\.\active
            \catcode`\,\active
            \catcode`\;\active
            \catcode`\:\active
            \catcode`\?\active
            \catcode`\!\active
            \catcode`\/\active
            \lccode`\~`\~
        }
    \makeatother

    \let\OriginalVerbatim=\Verbatim
    \makeatletter
    \renewcommand{\Verbatim}[1][1]{%
        %\parskip\z@skip
        \sbox\Wrappedcontinuationbox {\Wrappedcontinuationsymbol}%
        \sbox\Wrappedvisiblespacebox {\FV@SetupFont\Wrappedvisiblespace}%
        \def\FancyVerbFormatLine ##1{\hsize\linewidth
            \vtop{\raggedright\hyphenpenalty\z@\exhyphenpenalty\z@
                \doublehyphendemerits\z@\finalhyphendemerits\z@
                \strut ##1\strut}%
        }%
        % If the linebreak is at a space, the latter will be displayed as visible
        % space at end of first line, and a continuation symbol starts next line.
        % Stretch/shrink are however usually zero for typewriter font.
        \def\FV@Space {%
            \nobreak\hskip\z@ plus\fontdimen3\font minus\fontdimen4\font
            \discretionary{\copy\Wrappedvisiblespacebox}{\Wrappedafterbreak}
            {\kern\fontdimen2\font}%
        }%

        % Allow breaks at special characters using \PYG... macros.
        \Wrappedbreaksatspecials
        % Breaks at punctuation characters . , ; ? ! and / need catcode=\active
        \OriginalVerbatim[#1,codes*=\Wrappedbreaksatpunct]%
    }
    \makeatother

    % Exact colors from NB
    \definecolor{incolor}{HTML}{303F9F}
    \definecolor{outcolor}{HTML}{D84315}
    \definecolor{cellborder}{HTML}{CFCFCF}
    \definecolor{cellbackground}{HTML}{F7F7F7}

    % prompt
    \makeatletter
    \newcommand{\boxspacing}{\kern\kvtcb@left@rule\kern\kvtcb@boxsep}
    \makeatother
    \newcommand{\prompt}[4]{
        {\ttfamily\llap{{\color{#2}[#3]:\hspace{3pt}#4}}\vspace{-\baselineskip}}
    }
    

    
    % Prevent overflowing lines due to hard-to-break entities
    \sloppy
    % Setup hyperref package
    \hypersetup{
      breaklinks=true,  % so long urls are correctly broken across lines
      colorlinks=true,
      urlcolor=urlcolor,
      linkcolor=linkcolor,
      citecolor=citecolor,
      }
    % Slightly bigger margins than the latex defaults
    
    \geometry{verbose,tmargin=1in,bmargin=1in,lmargin=1in,rmargin=1in}
    
    

\begin{document}
    
    \maketitle
    
    

    
    \section{MISSING VALUES \& IMPORT
DATA}\label{missing-values-import-data}

    \subsection{CREATO DA GRANIERI JOELE, TASSONE LEONARDO, RAPHAEL RODRIGO
E RADISHA
WARNAKULASURIYA}\label{creato-da-granieri-joele-tassone-leonardo-raphael-rodrigo-e-radisha-warnakulasuriya}

    \subsection{MISSING VALUES}\label{missing-values}

    \subsubsection{Generazione e Visualizzazione di Dati Casuali con Pandas
e
NumPy}\label{generazione-e-visualizzazione-di-dati-casuali-con-pandas-e-numpy}

    \begin{tcolorbox}[breakable, size=fbox, boxrule=1pt, pad at break*=1mm,colback=cellbackground, colframe=cellborder]
\prompt{In}{incolor}{1}{\boxspacing}
\begin{Verbatim}[commandchars=\\\{\}]
\PY{k+kn}{import} \PY{n+nn}{pandas} \PY{k}{as} \PY{n+nn}{pd}
\PY{k+kn}{import} \PY{n+nn}{numpy} \PY{k}{as} \PY{n+nn}{np}
\PY{k+kn}{import} \PY{n+nn}{matplotlib}\PY{n+nn}{.}\PY{n+nn}{pyplot} \PY{k}{as} \PY{n+nn}{plt}
\PY{k+kn}{import} \PY{n+nn}{seaborn} \PY{k}{as} \PY{n+nn}{sns}
\PY{k+kn}{import} \PY{n+nn}{plotly}\PY{n+nn}{.}\PY{n+nn}{express} \PY{k}{as} \PY{n+nn}{px}

\PY{c+c1}{\PYZsh{} Genera dati casuali per l\PYZsq{}esplorazione}
\PY{n}{np}\PY{o}{.}\PY{n}{random}\PY{o}{.}\PY{n}{seed}\PY{p}{(}\PY{l+m+mi}{42}\PY{p}{)}
\PY{n}{data} \PY{o}{=} \PY{p}{\PYZob{}}
    \PY{l+s+s1}{\PYZsq{}}\PY{l+s+s1}{Età}\PY{l+s+s1}{\PYZsq{}}\PY{p}{:} \PY{n}{np}\PY{o}{.}\PY{n}{random}\PY{o}{.}\PY{n}{randint}\PY{p}{(}\PY{l+m+mi}{18}\PY{p}{,} \PY{l+m+mi}{70}\PY{p}{,} \PY{n}{size}\PY{o}{=}\PY{l+m+mi}{1000}\PY{p}{)}\PY{p}{,}
    \PY{l+s+s1}{\PYZsq{}}\PY{l+s+s1}{Genere}\PY{l+s+s1}{\PYZsq{}}\PY{p}{:} \PY{n}{np}\PY{o}{.}\PY{n}{random}\PY{o}{.}\PY{n}{choice}\PY{p}{(}\PY{p}{[}\PY{l+s+s1}{\PYZsq{}}\PY{l+s+s1}{Maschio}\PY{l+s+s1}{\PYZsq{}}\PY{p}{,} \PY{l+s+s1}{\PYZsq{}}\PY{l+s+s1}{Femmina}\PY{l+s+s1}{\PYZsq{}}\PY{p}{]}\PY{p}{,} \PY{n}{size}\PY{o}{=}\PY{l+m+mi}{1000}\PY{p}{)}\PY{p}{,}
    \PY{l+s+s1}{\PYZsq{}}\PY{l+s+s1}{Punteggio}\PY{l+s+s1}{\PYZsq{}}\PY{p}{:} \PY{n}{np}\PY{o}{.}\PY{n}{random}\PY{o}{.}\PY{n}{uniform}\PY{p}{(}\PY{l+m+mi}{0}\PY{p}{,} \PY{l+m+mi}{100}\PY{p}{,} \PY{n}{size}\PY{o}{=}\PY{l+m+mi}{1000}\PY{p}{)}\PY{p}{,}
     \PY{c+c1}{\PYZsh{}random normal esce il numero con la media più alta, invece quella più bassa ha poca possibilità di uscire}
    \PY{l+s+s1}{\PYZsq{}}\PY{l+s+s1}{Reddito}\PY{l+s+s1}{\PYZsq{}}\PY{p}{:} \PY{n}{np}\PY{o}{.}\PY{n}{random}\PY{o}{.}\PY{n}{normal}\PY{p}{(}\PY{l+m+mi}{50000}\PY{p}{,} \PY{l+m+mi}{15000}\PY{p}{,} \PY{n}{size}\PY{o}{=}\PY{l+m+mi}{1000}\PY{p}{)}
\PY{p}{\PYZcb{}}

\PY{n}{df} \PY{o}{=} \PY{n}{pd}\PY{o}{.}\PY{n}{DataFrame}\PY{p}{(}\PY{n}{data}\PY{p}{)}

\PY{c+c1}{\PYZsh{} Visualizza le prime righe del dataset}
\PY{n+nb}{print}\PY{p}{(}\PY{n}{df}\PY{o}{.}\PY{n}{head}\PY{p}{(}\PY{p}{)}\PY{p}{)}
\end{Verbatim}
\end{tcolorbox}

    \begin{Verbatim}[commandchars=\\\{\}]
   Età   Genere  Punteggio       Reddito
0   56  Maschio  85.120691  52915.764524
1   69  Maschio  49.514653  44702.505608
2   46  Maschio  48.058658  55077.257652
3   32  Femmina  59.240778  45568.978848
4   60  Maschio  82.468097  52526.914644
    \end{Verbatim}

    \begin{quote}
Il codice genera un DataFrame di pandas chiamato df con 1000 righe di
dati casuali. Questi dati includono:

`Età': un array di 1000 numeri interi casuali tra 18 e 70.

`Genere': un array di 1000 scelte casuali tra `Maschio' e `Femmina'.

`Punteggio': un array di 1000 numeri float casuali tra 0 e 100.

`Reddito': un array di 1000 numeri generati da una distribuzione normale
con media 50000 e deviazione standard 15000.
\end{quote}

\begin{quote}
La linea print(df.head()) stampa le prime 5 righe del DataFrame df.
Questo è utile per avere un'anteprima dei dati senza dover stampare
l'intero DataFrame.
\end{quote}

    \subsubsection{Creazione e Anteprima di un DataFrame Pandas con Dati
Casuali}\label{creazione-e-anteprima-di-un-dataframe-pandas-con-dati-casuali}

    \begin{tcolorbox}[breakable, size=fbox, boxrule=1pt, pad at break*=1mm,colback=cellbackground, colframe=cellborder]
\prompt{In}{incolor}{2}{\boxspacing}
\begin{Verbatim}[commandchars=\\\{\}]
\PY{k+kn}{import} \PY{n+nn}{pandas} \PY{k}{as} \PY{n+nn}{pd}

\PY{c+c1}{\PYZsh{} Dataset con dati mancanti rappresentati da None o NaN}
\PY{n}{dataset} \PY{o}{=} \PY{p}{[}
    \PY{p}{\PYZob{}}\PY{l+s+s2}{\PYZdq{}}\PY{l+s+s2}{età}\PY{l+s+s2}{\PYZdq{}}\PY{p}{:} \PY{l+m+mi}{25}\PY{p}{,} \PY{l+s+s2}{\PYZdq{}}\PY{l+s+s2}{punteggio}\PY{l+s+s2}{\PYZdq{}}\PY{p}{:} \PY{l+m+mi}{90}\PY{p}{,} \PY{l+s+s2}{\PYZdq{}}\PY{l+s+s2}{ammesso}\PY{l+s+s2}{\PYZdq{}}\PY{p}{:} \PY{l+m+mi}{1}\PY{p}{\PYZcb{}}\PY{p}{,}
    \PY{p}{\PYZob{}}\PY{l+s+s2}{\PYZdq{}}\PY{l+s+s2}{età}\PY{l+s+s2}{\PYZdq{}}\PY{p}{:} \PY{k+kc}{None}\PY{p}{,} \PY{l+s+s2}{\PYZdq{}}\PY{l+s+s2}{punteggio}\PY{l+s+s2}{\PYZdq{}}\PY{p}{:} \PY{l+m+mi}{85}\PY{p}{,} \PY{l+s+s2}{\PYZdq{}}\PY{l+s+s2}{ammesso}\PY{l+s+s2}{\PYZdq{}}\PY{p}{:} \PY{l+m+mi}{0}\PY{p}{\PYZcb{}}\PY{p}{,}
    \PY{p}{\PYZob{}}\PY{l+s+s2}{\PYZdq{}}\PY{l+s+s2}{età}\PY{l+s+s2}{\PYZdq{}}\PY{p}{:} \PY{l+m+mi}{28}\PY{p}{,} \PY{l+s+s2}{\PYZdq{}}\PY{l+s+s2}{punteggio}\PY{l+s+s2}{\PYZdq{}}\PY{p}{:} \PY{k+kc}{None}\PY{p}{,} \PY{l+s+s2}{\PYZdq{}}\PY{l+s+s2}{ammesso}\PY{l+s+s2}{\PYZdq{}}\PY{p}{:} \PY{l+m+mi}{1}\PY{p}{\PYZcb{}}\PY{p}{,}
    \PY{p}{\PYZob{}}\PY{l+s+s2}{\PYZdq{}}\PY{l+s+s2}{età}\PY{l+s+s2}{\PYZdq{}}\PY{p}{:} \PY{k+kc}{None}\PY{p}{,} \PY{l+s+s2}{\PYZdq{}}\PY{l+s+s2}{punteggio}\PY{l+s+s2}{\PYZdq{}}\PY{p}{:} \PY{l+m+mi}{75}\PY{p}{,} \PY{l+s+s2}{\PYZdq{}}\PY{l+s+s2}{ammesso}\PY{l+s+s2}{\PYZdq{}}\PY{p}{:} \PY{l+m+mi}{1}\PY{p}{\PYZcb{}}\PY{p}{,}
    \PY{p}{\PYZob{}}\PY{l+s+s2}{\PYZdq{}}\PY{l+s+s2}{età}\PY{l+s+s2}{\PYZdq{}}\PY{p}{:} \PY{l+m+mi}{23}\PY{p}{,} \PY{l+s+s2}{\PYZdq{}}\PY{l+s+s2}{punteggio}\PY{l+s+s2}{\PYZdq{}}\PY{p}{:} \PY{k+kc}{None}\PY{p}{,} \PY{l+s+s2}{\PYZdq{}}\PY{l+s+s2}{ammesso}\PY{l+s+s2}{\PYZdq{}}\PY{p}{:} \PY{k+kc}{None}\PY{p}{\PYZcb{}}\PY{p}{,}
    \PY{p}{\PYZob{}}\PY{l+s+s2}{\PYZdq{}}\PY{l+s+s2}{età}\PY{l+s+s2}{\PYZdq{}}\PY{p}{:} \PY{l+m+mi}{23}\PY{p}{,} \PY{l+s+s2}{\PYZdq{}}\PY{l+s+s2}{punteggio}\PY{l+s+s2}{\PYZdq{}}\PY{p}{:} \PY{l+m+mi}{77}\PY{p}{,} \PY{l+s+s2}{\PYZdq{}}\PY{l+s+s2}{ammesso}\PY{l+s+s2}{\PYZdq{}}\PY{p}{:} \PY{k+kc}{None}\PY{p}{\PYZcb{}}\PY{p}{,}
\PY{p}{]}
\PY{n}{df} \PY{o}{=} \PY{n}{pd}\PY{o}{.}\PY{n}{DataFrame}\PY{p}{(}\PY{n}{dataset}\PY{p}{)}
\PY{n}{df}
\end{Verbatim}
\end{tcolorbox}

            \begin{tcolorbox}[breakable, size=fbox, boxrule=.5pt, pad at break*=1mm, opacityfill=0]
\prompt{Out}{outcolor}{2}{\boxspacing}
\begin{Verbatim}[commandchars=\\\{\}]
    età  punteggio  ammesso
0  25.0       90.0      1.0
1   NaN       85.0      0.0
2  28.0        NaN      1.0
3   NaN       75.0      1.0
4  23.0        NaN      NaN
5  23.0       77.0      NaN
\end{Verbatim}
\end{tcolorbox}
        
    \begin{quote}
Il codice crea un DataFrame di pandas chiamato df da un elenco di
dizionari. Ogni dizionario rappresenta un record con campi `età',
`punteggio' e `ammesso'. Alcuni dei valori in questi campi sono None,
che pandas interpreta come valori mancanti o NaN (Not a Number).
\end{quote}

    \subsubsection{Accesso e Manipolazione di Dati in un DataFrame
Pandas}\label{accesso-e-manipolazione-di-dati-in-un-dataframe-pandas}

    \begin{tcolorbox}[breakable, size=fbox, boxrule=1pt, pad at break*=1mm,colback=cellbackground, colframe=cellborder]
\prompt{In}{incolor}{3}{\boxspacing}
\begin{Verbatim}[commandchars=\\\{\}]
\PY{n}{df}\PY{p}{[}\PY{l+s+s2}{\PYZdq{}}\PY{l+s+s2}{età}\PY{l+s+s2}{\PYZdq{}}\PY{p}{]}
\end{Verbatim}
\end{tcolorbox}

            \begin{tcolorbox}[breakable, size=fbox, boxrule=.5pt, pad at break*=1mm, opacityfill=0]
\prompt{Out}{outcolor}{3}{\boxspacing}
\begin{Verbatim}[commandchars=\\\{\}]
0    25.0
1     NaN
2    28.0
3     NaN
4    23.0
5    23.0
Name: età, dtype: float64
\end{Verbatim}
\end{tcolorbox}
        
    \begin{quote}
L'espressione df{[}``età''{]} seleziona la colonna ``età'' dal DataFrame
df. Questo restituirà una serie di pandas che contiene tutti i valori
della colonna ``età''. Se ci sono valori mancanti in questa colonna,
verranno rappresentati come NaN (Not a Number)
\end{quote}

    \subsubsection{Identificazione di Valori Mancanti in un DataFrame
Pandas}\label{identificazione-di-valori-mancanti-in-un-dataframe-pandas}

    \begin{tcolorbox}[breakable, size=fbox, boxrule=1pt, pad at break*=1mm,colback=cellbackground, colframe=cellborder]
\prompt{In}{incolor}{4}{\boxspacing}
\begin{Verbatim}[commandchars=\\\{\}]
\PY{k+kn}{import} \PY{n+nn}{pandas} \PY{k}{as} \PY{n+nn}{pd}
\PY{k+kn}{import} \PY{n+nn}{seaborn} \PY{k}{as} \PY{n+nn}{sns}
\PY{k+kn}{import} \PY{n+nn}{matplotlib}\PY{n+nn}{.}\PY{n+nn}{pyplot} \PY{k}{as} \PY{n+nn}{plt}
\PY{k+kn}{import} \PY{n+nn}{numpy} \PY{k}{as} \PY{n+nn}{np}

\PY{c+c1}{\PYZsh{} Genera dati di esempio}
\PY{n}{data} \PY{o}{=} \PY{p}{\PYZob{}}
    \PY{l+s+s1}{\PYZsq{}}\PY{l+s+s1}{Feature1}\PY{l+s+s1}{\PYZsq{}}\PY{p}{:} \PY{p}{[}\PY{l+m+mi}{1}\PY{p}{,} \PY{l+m+mi}{2}\PY{p}{,} \PY{n}{np}\PY{o}{.}\PY{n}{nan}\PY{p}{,} \PY{l+m+mi}{4}\PY{p}{,} \PY{l+m+mi}{5}\PY{p}{]}\PY{p}{,}
    \PY{l+s+s1}{\PYZsq{}}\PY{l+s+s1}{Feature2}\PY{l+s+s1}{\PYZsq{}}\PY{p}{:} \PY{p}{[}\PY{n}{np}\PY{o}{.}\PY{n}{nan}\PY{p}{,} \PY{l+m+mi}{2}\PY{p}{,} \PY{l+m+mi}{3}\PY{p}{,} \PY{l+m+mi}{4}\PY{p}{,} \PY{n}{np}\PY{o}{.}\PY{n}{nan}\PY{p}{]}\PY{p}{,}
    \PY{l+s+s1}{\PYZsq{}}\PY{l+s+s1}{Feature3}\PY{l+s+s1}{\PYZsq{}}\PY{p}{:} \PY{p}{[}\PY{l+m+mi}{1}\PY{p}{,} \PY{n}{np}\PY{o}{.}\PY{n}{nan}\PY{p}{,} \PY{l+m+mi}{3}\PY{p}{,} \PY{l+m+mi}{4}\PY{p}{,} \PY{l+m+mi}{5}\PY{p}{]}
\PY{p}{\PYZcb{}}

\PY{c+c1}{\PYZsh{} Crea un DataFrame}
\PY{n}{df} \PY{o}{=} \PY{n}{pd}\PY{o}{.}\PY{n}{DataFrame}\PY{p}{(}\PY{n}{data}\PY{p}{)}

\PY{c+c1}{\PYZsh{} Calcola la matrice di missing values}
\PY{n}{missing\PYZus{}matrix} \PY{o}{=} \PY{n}{df}\PY{o}{.}\PY{n}{isnull}\PY{p}{(}\PY{p}{)}
\PY{n}{missing\PYZus{}matrix}
\end{Verbatim}
\end{tcolorbox}

            \begin{tcolorbox}[breakable, size=fbox, boxrule=.5pt, pad at break*=1mm, opacityfill=0]
\prompt{Out}{outcolor}{4}{\boxspacing}
\begin{Verbatim}[commandchars=\\\{\}]
   Feature1  Feature2  Feature3
0     False      True     False
1     False     False      True
2      True     False     False
3     False     False     False
4     False      True     False
\end{Verbatim}
\end{tcolorbox}
        
    \begin{quote}
Il comando missing\_matrix = df.isnull() crea una nuova variabile
chiamata missing\_matrix che è un DataFrame di stesse dimensioni di df.
In missing\_matrix, ogni elemento è un valore booleano che indica se
l'elemento corrispondente in df è un valore mancante (True) o no
(False). In questo DataFrame, True indica un valore mancante e False
indica un valore non mancante.
\end{quote}

    \subsubsection{Selezione di Righe con Valori Mancanti in un DataFrame
Pandas}\label{selezione-di-righe-con-valori-mancanti-in-un-dataframe-pandas}

    \begin{tcolorbox}[breakable, size=fbox, boxrule=1pt, pad at break*=1mm,colback=cellbackground, colframe=cellborder]
\prompt{In}{incolor}{5}{\boxspacing}
\begin{Verbatim}[commandchars=\\\{\}]
\PY{n}{righe\PYZus{}con\PYZus{}dati\PYZus{}mancanti} \PY{o}{=} \PY{n}{df}\PY{p}{[}\PY{n}{df}\PY{o}{.}\PY{n}{isnull}\PY{p}{(}\PY{p}{)}\PY{o}{.}\PY{n}{any}\PY{p}{(}\PY{n}{axis}\PY{o}{=}\PY{l+m+mi}{1}\PY{p}{)}\PY{p}{]}
\PY{n}{righe\PYZus{}con\PYZus{}dati\PYZus{}mancanti}
\end{Verbatim}
\end{tcolorbox}

            \begin{tcolorbox}[breakable, size=fbox, boxrule=.5pt, pad at break*=1mm, opacityfill=0]
\prompt{Out}{outcolor}{5}{\boxspacing}
\begin{Verbatim}[commandchars=\\\{\}]
   Feature1  Feature2  Feature3
0       1.0       NaN       1.0
1       2.0       2.0       NaN
2       NaN       3.0       3.0
4       5.0       NaN       5.0
\end{Verbatim}
\end{tcolorbox}
        
    \begin{quote}
Il codice seleziona tutte le righe del DataFrame df che contengono
almeno un valore mancante (NaN o None) e le assegna alla variabile
righe\_con\_dati\_mancanti.

Il comando df.isnull().any(axis=1) restituisce una serie booleana che
indica se ciascuna riga ha almeno un valore mancante. Quindi,
df{[}df.isnull().any(axis=1){]} seleziona solo le righe di df per le
quali il valore corrispondente nella serie booleana è True.
\end{quote}

    \subsubsection{Calcolo del Numero Totale di Righe con Dati Mancanti in
un DataFrame
Pandas}\label{calcolo-del-numero-totale-di-righe-con-dati-mancanti-in-un-dataframe-pandas}

    \begin{tcolorbox}[breakable, size=fbox, boxrule=1pt, pad at break*=1mm,colback=cellbackground, colframe=cellborder]
\prompt{In}{incolor}{6}{\boxspacing}
\begin{Verbatim}[commandchars=\\\{\}]
\PY{n}{totale\PYZus{}dati\PYZus{}mancanti} \PY{o}{=} \PY{n}{righe\PYZus{}con\PYZus{}dati\PYZus{}mancanti}\PY{o}{.}\PY{n}{shape}\PY{p}{[}\PY{l+m+mi}{0}\PY{p}{]}
\PY{n}{totale\PYZus{}dati\PYZus{}mancanti}
\end{Verbatim}
\end{tcolorbox}

            \begin{tcolorbox}[breakable, size=fbox, boxrule=.5pt, pad at break*=1mm, opacityfill=0]
\prompt{Out}{outcolor}{6}{\boxspacing}
\begin{Verbatim}[commandchars=\\\{\}]
4
\end{Verbatim}
\end{tcolorbox}
        
    \begin{quote}
Il codice calcola il numero totale di righe con dati mancanti nel
DataFrame df e lo assegna alla variabile totale\_dati\_mancanti.

La funzione shape{[}0{]} restituisce il numero di righe in un DataFrame.
Quindi, righe\_con\_dati\_mancanti.shape{[}0{]} restituisce il numero di
righe nel DataFrame righe\_con\_dati\_mancanti, che contiene solo le
righe di df con almeno un valore mancante.

Il valore di totale\_dati\_mancanti sarà quindi il numero totale di
righe con almeno un valore mancante nel DataFrame originale df.
\end{quote}

    \subsubsection{Calcolo e Stampa di Dati Mancanti nel
Dataset}\label{calcolo-e-stampa-di-dati-mancanti-nel-dataset}

    \begin{tcolorbox}[breakable, size=fbox, boxrule=1pt, pad at break*=1mm,colback=cellbackground, colframe=cellborder]
\prompt{In}{incolor}{7}{\boxspacing}
\begin{Verbatim}[commandchars=\\\{\}]
\PY{n+nb}{print}\PY{p}{(}\PY{l+s+s2}{\PYZdq{}}\PY{l+s+s2}{righe con dati mancanti:}\PY{l+s+s2}{\PYZdq{}}\PY{p}{)}
\PY{n+nb}{print}\PY{p}{(}\PY{n}{righe\PYZus{}con\PYZus{}dati\PYZus{}mancanti}\PY{p}{)}
\PY{n+nb}{print}\PY{p}{(}\PY{l+s+s2}{\PYZdq{}}\PY{l+s+s2}{totale dati mancanti: }\PY{l+s+s2}{\PYZdq{}}\PY{p}{,} \PY{n}{totale\PYZus{}dati\PYZus{}mancanti}\PY{p}{)}
\end{Verbatim}
\end{tcolorbox}

    \begin{Verbatim}[commandchars=\\\{\}]
righe con dati mancanti:
   Feature1  Feature2  Feature3
0       1.0       NaN       1.0
1       2.0       2.0       NaN
2       NaN       3.0       3.0
4       5.0       NaN       5.0
totale dati mancanti:  4
    \end{Verbatim}

    \begin{quote}
Il codice stampa le righe del DataFrame df che contengono almeno un
valore mancante (NaN o None), e poi stampa il numero totale di queste
righe.
\end{quote}

    \subsubsection{Creazione e Visualizzazione di un DataFrame con Dati
Mancanti}\label{creazione-e-visualizzazione-di-un-dataframe-con-dati-mancanti}

    \begin{tcolorbox}[breakable, size=fbox, boxrule=1pt, pad at break*=1mm,colback=cellbackground, colframe=cellborder]
\prompt{In}{incolor}{8}{\boxspacing}
\begin{Verbatim}[commandchars=\\\{\}]
\PY{k+kn}{import} \PY{n+nn}{pandas} \PY{k}{as} \PY{n+nn}{pd}

\PY{c+c1}{\PYZsh{} Dataset con dati mancanti rappresentati da None o NaN}
\PY{n}{dataset} \PY{o}{=} \PY{p}{[}
    \PY{p}{\PYZob{}}\PY{l+s+s2}{\PYZdq{}}\PY{l+s+s2}{nome}\PY{l+s+s2}{\PYZdq{}}\PY{p}{:} \PY{l+s+s2}{\PYZdq{}}\PY{l+s+s2}{Alice}\PY{l+s+s2}{\PYZdq{}}\PY{p}{,} \PY{l+s+s2}{\PYZdq{}}\PY{l+s+s2}{età}\PY{l+s+s2}{\PYZdq{}}\PY{p}{:} \PY{l+m+mi}{25}\PY{p}{,} \PY{l+s+s2}{\PYZdq{}}\PY{l+s+s2}{punteggio}\PY{l+s+s2}{\PYZdq{}}\PY{p}{:} \PY{l+m+mi}{90}\PY{p}{,} \PY{l+s+s2}{\PYZdq{}}\PY{l+s+s2}{email}\PY{l+s+s2}{\PYZdq{}}\PY{p}{:} \PY{l+s+s2}{\PYZdq{}}\PY{l+s+s2}{alice@email.com}\PY{l+s+s2}{\PYZdq{}}\PY{p}{\PYZcb{}}\PY{p}{,}
    \PY{p}{\PYZob{}}\PY{l+s+s2}{\PYZdq{}}\PY{l+s+s2}{nome}\PY{l+s+s2}{\PYZdq{}}\PY{p}{:} \PY{l+s+s2}{\PYZdq{}}\PY{l+s+s2}{Bob}\PY{l+s+s2}{\PYZdq{}}\PY{p}{,} \PY{l+s+s2}{\PYZdq{}}\PY{l+s+s2}{età}\PY{l+s+s2}{\PYZdq{}}\PY{p}{:} \PY{l+m+mi}{22}\PY{p}{,} \PY{l+s+s2}{\PYZdq{}}\PY{l+s+s2}{punteggio}\PY{l+s+s2}{\PYZdq{}}\PY{p}{:} \PY{k+kc}{None}\PY{p}{,} \PY{l+s+s2}{\PYZdq{}}\PY{l+s+s2}{email}\PY{l+s+s2}{\PYZdq{}}\PY{p}{:} \PY{k+kc}{None}\PY{p}{\PYZcb{}}\PY{p}{,}
    \PY{p}{\PYZob{}}\PY{l+s+s2}{\PYZdq{}}\PY{l+s+s2}{nome}\PY{l+s+s2}{\PYZdq{}}\PY{p}{:} \PY{l+s+s2}{\PYZdq{}}\PY{l+s+s2}{Charlie}\PY{l+s+s2}{\PYZdq{}}\PY{p}{,} \PY{l+s+s2}{\PYZdq{}}\PY{l+s+s2}{età}\PY{l+s+s2}{\PYZdq{}}\PY{p}{:} \PY{l+m+mi}{28}\PY{p}{,} \PY{l+s+s2}{\PYZdq{}}\PY{l+s+s2}{punteggio}\PY{l+s+s2}{\PYZdq{}}\PY{p}{:} \PY{l+m+mi}{75}\PY{p}{,} \PY{l+s+s2}{\PYZdq{}}\PY{l+s+s2}{email}\PY{l+s+s2}{\PYZdq{}}\PY{p}{:} \PY{l+s+s2}{\PYZdq{}}\PY{l+s+s2}{charlie@email.com}\PY{l+s+s2}{\PYZdq{}}\PY{p}{\PYZcb{}}\PY{p}{,}
\PY{p}{]}

\PY{c+c1}{\PYZsh{} Converti il dataset in un DataFrame}
\PY{n}{df} \PY{o}{=} \PY{n}{pd}\PY{o}{.}\PY{n}{DataFrame}\PY{p}{(}\PY{n}{dataset}\PY{p}{)}
\PY{n}{df}
\end{Verbatim}
\end{tcolorbox}

            \begin{tcolorbox}[breakable, size=fbox, boxrule=.5pt, pad at break*=1mm, opacityfill=0]
\prompt{Out}{outcolor}{8}{\boxspacing}
\begin{Verbatim}[commandchars=\\\{\}]
      nome  età  punteggio              email
0    Alice   25       90.0    alice@email.com
1      Bob   22        NaN               None
2  Charlie   28       75.0  charlie@email.com
\end{Verbatim}
\end{tcolorbox}
        
    \begin{quote}
Il codice fa quanto segue:

Importa la libreria pandas con l'alias pd.

Crea un dataset, che è una lista di dizionari. Ogni dizionario
rappresenta una riga di dati con quattro attributi: ``nome'', ``età'',
``punteggio'' e ``email''. Alcuni valori sono None, che rappresenta dati
mancanti.

Converte il dataset in un DataFrame di pandas usando
pd.DataFrame(dataset). Un DataFrame è una struttura dati bidimensionale,
simile a una tabella, con righe e colonne.

Infine, il DataFrame df viene stampato.
\end{quote}

\begin{quote}
Quindi, il codice crea un DataFrame da un dataset e lo stampa.
\end{quote}

    \subsubsection{Rimozione delle Righe con Dati Mancanti dal
DataFrame}\label{rimozione-delle-righe-con-dati-mancanti-dal-dataframe}

    \begin{tcolorbox}[breakable, size=fbox, boxrule=1pt, pad at break*=1mm,colback=cellbackground, colframe=cellborder]
\prompt{In}{incolor}{9}{\boxspacing}
\begin{Verbatim}[commandchars=\\\{\}]
\PY{n}{df1}\PY{o}{=}\PY{n}{df}\PY{o}{.}\PY{n}{dropna}\PY{p}{(}\PY{n}{inplace}\PY{o}{=}\PY{k+kc}{False}\PY{p}{)}
\PY{n}{df1}
\end{Verbatim}
\end{tcolorbox}

            \begin{tcolorbox}[breakable, size=fbox, boxrule=.5pt, pad at break*=1mm, opacityfill=0]
\prompt{Out}{outcolor}{9}{\boxspacing}
\begin{Verbatim}[commandchars=\\\{\}]
      nome  età  punteggio              email
0    Alice   25       90.0    alice@email.com
2  Charlie   28       75.0  charlie@email.com
\end{Verbatim}
\end{tcolorbox}
        
    \begin{quote}
Il codice crea un nuovo DataFrame df1 rimuovendo tutte le righe con dati
mancanti (None o NaN) dal DataFrame originale df. L'argomento
inplace=False significa che la modifica non viene applicata al DataFrame
originale df, ma viene restituito un nuovo DataFrame con le modifiche.
Quindi, df1 sarà un DataFrame che non contiene righe con dati mancanti.
\end{quote}

    \subsubsection{Creazione e Visualizzazione di una Mappa Termica dei
Valori
Mancanti}\label{creazione-e-visualizzazione-di-una-mappa-termica-dei-valori-mancanti}

    \begin{tcolorbox}[breakable, size=fbox, boxrule=1pt, pad at break*=1mm,colback=cellbackground, colframe=cellborder]
\prompt{In}{incolor}{10}{\boxspacing}
\begin{Verbatim}[commandchars=\\\{\}]
\PY{c+c1}{\PYZsh{} Crea una heatmap colorata}
\PY{n}{plt}\PY{o}{.}\PY{n}{figure}\PY{p}{(}\PY{n}{figsize}\PY{o}{=}\PY{p}{(}\PY{l+m+mi}{8}\PY{p}{,} \PY{l+m+mi}{6}\PY{p}{)}\PY{p}{)}
\PY{c+c1}{\PYZsh{}cbar serve per una barra di clore, False non lo voglio, Truese lo voglio}
\PY{n}{sns}\PY{o}{.}\PY{n}{heatmap}\PY{p}{(}\PY{n}{missing\PYZus{}matrix}\PY{p}{,} \PY{n}{cmap}\PY{o}{=}\PY{l+s+s1}{\PYZsq{}}\PY{l+s+s1}{viridis}\PY{l+s+s1}{\PYZsq{}}\PY{p}{,} \PY{n}{cbar}\PY{o}{=}\PY{k+kc}{True}\PY{p}{,}\PY{n}{alpha}\PY{o}{=}\PY{l+m+mf}{0.8}\PY{p}{)}
\PY{n}{plt}\PY{o}{.}\PY{n}{title}\PY{p}{(}\PY{l+s+s1}{\PYZsq{}}\PY{l+s+s1}{Matrice di Missing Values}\PY{l+s+s1}{\PYZsq{}}\PY{p}{)}
\PY{n}{plt}\PY{o}{.}\PY{n}{show}
\end{Verbatim}
\end{tcolorbox}

            \begin{tcolorbox}[breakable, size=fbox, boxrule=.5pt, pad at break*=1mm, opacityfill=0]
\prompt{Out}{outcolor}{10}{\boxspacing}
\begin{Verbatim}[commandchars=\\\{\}]
<function matplotlib.pyplot.show(close=None, block=None)>
\end{Verbatim}
\end{tcolorbox}
        
    \begin{center}
    \adjustimage{max size={0.9\linewidth}{0.9\paperheight}}{output_31_1.png}
    \end{center}
    { \hspace*{\fill} \\}
    
    \begin{quote}
Il codice fa quanto segue:

Crea una nuova figura con dimensioni specificate (8x6).

Utilizza la funzione heatmap della libreria seaborn (alias sns) per
creare una mappa termica (heatmap) della matrice missing\_matrix. La
mappa termica è colorata usando il colormap `viridis'. L'argomento
cbar=True indica che vuoi visualizzare una barra di colore che mostra la
scala dei colori utilizzati nella mappa termica. L'argomento alpha=0.8
imposta l'opacità dei colori nella mappa termica.

Imposta il titolo della mappa termica come `Matrice di Missing Values'.

Infine, il comando plt.show (che dovrebbe essere plt.show()) viene
utilizzato per visualizzare la figura.
\end{quote}

\begin{quote}
Quindi, il codice crea e visualizza una mappa termica colorata della
matrice missing\_matrix, che rappresenta i valori mancanti nel
DataFrame.
\end{quote}

    \subsubsection{Creazione di DataFrame con Dati Specificati e DataFrame
Vuoto}\label{creazione-di-dataframe-con-dati-specificati-e-dataframe-vuoto}

    \begin{tcolorbox}[breakable, size=fbox, boxrule=1pt, pad at break*=1mm,colback=cellbackground, colframe=cellborder]
\prompt{In}{incolor}{11}{\boxspacing}
\begin{Verbatim}[commandchars=\\\{\}]
\PY{k+kn}{import} \PY{n+nn}{pandas} \PY{k}{as} \PY{n+nn}{pd}
\PY{k+kn}{import} \PY{n+nn}{seaborn} \PY{k}{as} \PY{n+nn}{sns}
\PY{k+kn}{import} \PY{n+nn}{numpy} \PY{k}{as} \PY{n+nn}{np}
\PY{k+kn}{import} \PY{n+nn}{matplotlib}\PY{n+nn}{.}\PY{n+nn}{pyplot} \PY{k}{as} \PY{n+nn}{plt}

\PY{c+c1}{\PYZsh{} Genera dati di esempio}
\PY{n}{data} \PY{o}{=} \PY{p}{\PYZob{}}
    \PY{l+s+s1}{\PYZsq{}}\PY{l+s+s1}{Variable1}\PY{l+s+s1}{\PYZsq{}}\PY{p}{:} \PY{p}{[}\PY{l+m+mi}{1}\PY{p}{,} \PY{l+m+mi}{2}\PY{p}{,} \PY{l+m+mi}{3}\PY{p}{,} \PY{l+m+mi}{4}\PY{p}{,} \PY{l+m+mi}{5}\PY{p}{]}\PY{p}{,}
    \PY{l+s+s1}{\PYZsq{}}\PY{l+s+s1}{Variable2}\PY{l+s+s1}{\PYZsq{}}\PY{p}{:} \PY{p}{[}\PY{l+m+mi}{1}\PY{p}{,} \PY{l+m+mi}{2}\PY{p}{,} \PY{n}{np}\PY{o}{.}\PY{n}{nan}\PY{p}{,} \PY{l+m+mi}{4}\PY{p}{,} \PY{n}{np}\PY{o}{.}\PY{n}{nan}\PY{p}{]}\PY{p}{,}
    \PY{l+s+s1}{\PYZsq{}}\PY{l+s+s1}{Missing\PYZus{}Column}\PY{l+s+s1}{\PYZsq{}}\PY{p}{:} \PY{p}{[}\PY{l+s+s1}{\PYZsq{}}\PY{l+s+s1}{A}\PY{l+s+s1}{\PYZsq{}}\PY{p}{,} \PY{l+s+s1}{\PYZsq{}}\PY{l+s+s1}{B}\PY{l+s+s1}{\PYZsq{}}\PY{p}{,} \PY{l+s+s1}{\PYZsq{}}\PY{l+s+s1}{A}\PY{l+s+s1}{\PYZsq{}}\PY{p}{,} \PY{l+s+s1}{\PYZsq{}}\PY{l+s+s1}{C}\PY{l+s+s1}{\PYZsq{}}\PY{p}{,} \PY{n}{np}\PY{o}{.}\PY{n}{nan}\PY{p}{]}
\PY{p}{\PYZcb{}}
\PY{c+c1}{\PYZsh{} Crea un DataFrame}
\PY{n}{df} \PY{o}{=} \PY{n}{pd}\PY{o}{.}\PY{n}{DataFrame}\PY{p}{(}\PY{n}{data}\PY{p}{)}
\PY{n}{df1}\PY{o}{=}\PY{n}{pd}\PY{o}{.}\PY{n}{DataFrame}\PY{p}{(}\PY{p}{)}
\PY{n}{df}
\end{Verbatim}
\end{tcolorbox}

            \begin{tcolorbox}[breakable, size=fbox, boxrule=.5pt, pad at break*=1mm, opacityfill=0]
\prompt{Out}{outcolor}{11}{\boxspacing}
\begin{Verbatim}[commandchars=\\\{\}]
   Variable1  Variable2 Missing\_Column
0          1        1.0              A
1          2        2.0              B
2          3        NaN              A
3          4        4.0              C
4          5        NaN            NaN
\end{Verbatim}
\end{tcolorbox}
        
    \begin{quote}
Il codice fa quanto segue:

Importa le librerie necessarie: pandas (come pd), seaborn (come sns),
numpy (come np) e matplotlib.pyplot (come plt).

Crea un dizionario chiamato data con tre chiavi: `Variable1',
`Variable2' e `Missing\_Column'. Ogni chiave ha come valore una lista di
cinque elementi, alcuni dei quali sono np.nan, che rappresenta un valore
mancante.

Crea un DataFrame df da data utilizzando pd.DataFrame(data).

Crea un altro DataFrame vuoto df1.

Infine, stampa il DataFrame df.
\end{quote}

\begin{quote}
Quindi, il codice crea due DataFrame, uno con i dati specificati e uno
vuoto, e stampa il DataFrame con i dati.
\end{quote}

    \subsubsection{Selezione delle Colonne Numeriche dal
DataFrame}\label{selezione-delle-colonne-numeriche-dal-dataframe}

    \begin{tcolorbox}[breakable, size=fbox, boxrule=1pt, pad at break*=1mm,colback=cellbackground, colframe=cellborder]
\prompt{In}{incolor}{12}{\boxspacing}
\begin{Verbatim}[commandchars=\\\{\}]
\PY{n}{numeric\PYZus{}cols} \PY{o}{=} \PY{n}{df}\PY{o}{.}\PY{n}{select\PYZus{}dtypes}\PY{p}{(}\PY{n}{include}\PY{o}{=}\PY{p}{[}\PY{l+s+s1}{\PYZsq{}}\PY{l+s+s1}{number}\PY{l+s+s1}{\PYZsq{}}\PY{p}{]}\PY{p}{)}
\PY{n}{numeric\PYZus{}cols}
\end{Verbatim}
\end{tcolorbox}

            \begin{tcolorbox}[breakable, size=fbox, boxrule=.5pt, pad at break*=1mm, opacityfill=0]
\prompt{Out}{outcolor}{12}{\boxspacing}
\begin{Verbatim}[commandchars=\\\{\}]
   Variable1  Variable2
0          1        1.0
1          2        2.0
2          3        NaN
3          4        4.0
4          5        NaN
\end{Verbatim}
\end{tcolorbox}
        
    \begin{quote}
Il codice seleziona tutte le colonne nel DataFrame df che contengono
dati numerici. Questo viene fatto utilizzando il metodo select\_dtypes
con l'argomento include={[}`number'{]}. Il risultato è un nuovo
DataFrame numeric\_cols che contiene solo le colonne con dati numerici
dal DataFrame originale df.
\end{quote}

    \subsubsection{Selezione e Stampa dei Nomi delle Colonne Numeriche dal
DataFrame}\label{selezione-e-stampa-dei-nomi-delle-colonne-numeriche-dal-dataframe}

    \begin{tcolorbox}[breakable, size=fbox, boxrule=1pt, pad at break*=1mm,colback=cellbackground, colframe=cellborder]
\prompt{In}{incolor}{13}{\boxspacing}
\begin{Verbatim}[commandchars=\\\{\}]
\PY{n}{numeric\PYZus{}cols} \PY{o}{=} \PY{n}{df}\PY{o}{.}\PY{n}{select\PYZus{}dtypes}\PY{p}{(}\PY{n}{include}\PY{o}{=}\PY{p}{[}\PY{l+s+s1}{\PYZsq{}}\PY{l+s+s1}{number}\PY{l+s+s1}{\PYZsq{}}\PY{p}{]}\PY{p}{)}
\PY{n}{numeric\PYZus{}cols}\PY{o}{.}\PY{n}{columns}
\end{Verbatim}
\end{tcolorbox}

            \begin{tcolorbox}[breakable, size=fbox, boxrule=.5pt, pad at break*=1mm, opacityfill=0]
\prompt{Out}{outcolor}{13}{\boxspacing}
\begin{Verbatim}[commandchars=\\\{\}]
Index(['Variable1', 'Variable2'], dtype='object')
\end{Verbatim}
\end{tcolorbox}
        
    \begin{quote}
Il codice fa quanto segue:

Seleziona tutte le colonne nel DataFrame df che contengono dati numerici
utilizzando il metodo select\_dtypes con l'argomento
include={[}`number'{]}. Il risultato è un nuovo DataFrame numeric\_cols
che contiene solo le colonne con dati numerici dal DataFrame originale
df.

Infine, stampa i nomi delle colonne del DataFrame numeric\_cols
utilizzando l'attributo columns.
\end{quote}

\begin{quote}
Quindi, il codice seleziona le colonne numeriche dal DataFrame df e
stampa i loro nomi.
\end{quote}

    \subsubsection{Sostituzione dei Valori Mancanti con la Media nelle
Colonne Numeriche del
DataFrame}\label{sostituzione-dei-valori-mancanti-con-la-media-nelle-colonne-numeriche-del-dataframe}

    \begin{tcolorbox}[breakable, size=fbox, boxrule=1pt, pad at break*=1mm,colback=cellbackground, colframe=cellborder]
\prompt{In}{incolor}{14}{\boxspacing}
\begin{Verbatim}[commandchars=\\\{\}]
\PY{n}{df1}\PY{p}{[}\PY{n}{numeric\PYZus{}cols}\PY{o}{.}\PY{n}{columns}\PY{p}{]} \PY{o}{=} \PY{n}{df}\PY{p}{[}\PY{n}{numeric\PYZus{}cols}\PY{o}{.}\PY{n}{columns}\PY{p}{]}\PY{o}{.}\PY{n}{fillna}\PY{p}{(}\PY{n}{df}\PY{p}{[}\PY{n}{numeric\PYZus{}cols}\PY{o}{.}\PY{n}{columns}\PY{p}{]}\PY{o}{.}\PY{n}{mean}\PY{p}{(}\PY{p}{)}\PY{p}{)}
\PY{n}{df1}

\PY{c+c1}{\PYZsh{}crea colonne con stessi nomi (var1 e var2) = assegna valori delle colonne df originale(mean=media)}
\end{Verbatim}
\end{tcolorbox}

            \begin{tcolorbox}[breakable, size=fbox, boxrule=.5pt, pad at break*=1mm, opacityfill=0]
\prompt{Out}{outcolor}{14}{\boxspacing}
\begin{Verbatim}[commandchars=\\\{\}]
   Variable1  Variable2
0          1   1.000000
1          2   2.000000
2          3   2.333333
3          4   4.000000
4          5   2.333333
\end{Verbatim}
\end{tcolorbox}
        
    \begin{quote}
Il codice fa quanto segue:

Seleziona le colonne numeriche dal DataFrame originale df utilizzando
numeric\_cols.columns.

Utilizza il metodo fillna per riempire i valori mancanti (NaN) nelle
colonne numeriche del DataFrame df con la media (mean) dei valori
esistenti in ciascuna colonna.

Assegna il risultato al DataFrame df1, creando colonne con gli stessi
nomi delle colonne numeriche in df e assegnando i valori calcolati.

Infine, stampa il DataFrame df1.
\end{quote}

\begin{quote}
Quindi, il codice crea un nuovo DataFrame df1 che ha le stesse colonne
numeriche del DataFrame df, ma senza valori mancanti. I valori mancanti
sono sostituiti con la media dei valori esistenti per ciascuna colonna.
\end{quote}

    \subsubsection{Selezione e Stampa dei Nomi delle Colonne Categoriche dal
DataFrame}\label{selezione-e-stampa-dei-nomi-delle-colonne-categoriche-dal-dataframe}

    \begin{tcolorbox}[breakable, size=fbox, boxrule=1pt, pad at break*=1mm,colback=cellbackground, colframe=cellborder]
\prompt{In}{incolor}{15}{\boxspacing}
\begin{Verbatim}[commandchars=\\\{\}]
\PY{c+c1}{\PYZsh{}trattamento dei missing values nelle variabili categoriche}
\PY{n}{categorical\PYZus{}cols} \PY{o}{=} \PY{n}{df}\PY{o}{.}\PY{n}{select\PYZus{}dtypes}\PY{p}{(}\PY{n}{exclude}\PY{o}{=}\PY{p}{[}\PY{l+s+s1}{\PYZsq{}}\PY{l+s+s1}{number}\PY{l+s+s1}{\PYZsq{}}\PY{p}{]}\PY{p}{)} \PY{c+c1}{\PYZsh{}esclude colonne numeriche e fa vedere solo categoriche}
\PY{n}{categorical\PYZus{}cols}\PY{o}{.}\PY{n}{columns}
\end{Verbatim}
\end{tcolorbox}

            \begin{tcolorbox}[breakable, size=fbox, boxrule=.5pt, pad at break*=1mm, opacityfill=0]
\prompt{Out}{outcolor}{15}{\boxspacing}
\begin{Verbatim}[commandchars=\\\{\}]
Index(['Missing\_Column'], dtype='object')
\end{Verbatim}
\end{tcolorbox}
        
    \begin{quote}
Il codice fa quanto segue:

Seleziona tutte le colonne nel DataFrame df che non contengono dati
numerici utilizzando il metodo select\_dtypes con l'argomento
exclude={[}`number'{]}. Questo restituisce un nuovo DataFrame
categorical\_cols che contiene solo le colonne con dati categorici dal
DataFrame originale df.

Infine, stampa i nomi delle colonne del DataFrame categorical\_cols
utilizzando l'attributo columns.
\end{quote}

\begin{quote}
Quindi, il codice seleziona le colonne categoriche dal DataFrame df e
stampa i loro nomi.
\end{quote}

    \subsubsection{Calcolo del Numero Totale di Valori Mancanti per Colonna
nel
DataFrame}\label{calcolo-del-numero-totale-di-valori-mancanti-per-colonna-nel-dataframe}

    \begin{tcolorbox}[breakable, size=fbox, boxrule=1pt, pad at break*=1mm,colback=cellbackground, colframe=cellborder]
\prompt{In}{incolor}{16}{\boxspacing}
\begin{Verbatim}[commandchars=\\\{\}]
\PY{n}{df}\PY{o}{.}\PY{n}{isnull}\PY{p}{(}\PY{p}{)}\PY{o}{.}\PY{n}{sum}\PY{p}{(}\PY{p}{)}
\end{Verbatim}
\end{tcolorbox}

            \begin{tcolorbox}[breakable, size=fbox, boxrule=.5pt, pad at break*=1mm, opacityfill=0]
\prompt{Out}{outcolor}{16}{\boxspacing}
\begin{Verbatim}[commandchars=\\\{\}]
Variable1         0
Variable2         2
Missing\_Column    1
dtype: int64
\end{Verbatim}
\end{tcolorbox}
        
    \begin{quote}
Il codice calcola il numero totale di valori mancanti (NaN) per ogni
colonna nel DataFrame df. Questo viene fatto utilizzando il metodo
isnull() per identificare i valori mancanti, seguito dal metodo sum()
per sommare il numero di valori mancanti per ogni colonna. Il risultato
sarà una serie in cui l'indice è il nome della colonna e il valore è il
numero totale di valori mancanti in quella colonna.
\end{quote}

    \subsubsection{Calcolo della Percentuale di Valori Mancanti per Colonna
nel
DataFrame}\label{calcolo-della-percentuale-di-valori-mancanti-per-colonna-nel-dataframe}

    \begin{tcolorbox}[breakable, size=fbox, boxrule=1pt, pad at break*=1mm,colback=cellbackground, colframe=cellborder]
\prompt{In}{incolor}{17}{\boxspacing}
\begin{Verbatim}[commandchars=\\\{\}]
\PY{n}{missing\PYZus{}percent} \PY{o}{=} \PY{n}{df}\PY{o}{.}\PY{n}{isnull}\PY{p}{(}\PY{p}{)}\PY{o}{.}\PY{n}{sum}\PY{p}{(}\PY{p}{)} \PY{o}{/} \PY{n+nb}{len}\PY{p}{(}\PY{n}{df}\PY{p}{)} \PY{o}{*} \PY{l+m+mi}{100}
\PY{n}{missing\PYZus{}percent}
\end{Verbatim}
\end{tcolorbox}

            \begin{tcolorbox}[breakable, size=fbox, boxrule=.5pt, pad at break*=1mm, opacityfill=0]
\prompt{Out}{outcolor}{17}{\boxspacing}
\begin{Verbatim}[commandchars=\\\{\}]
Variable1          0.0
Variable2         40.0
Missing\_Column    20.0
dtype: float64
\end{Verbatim}
\end{tcolorbox}
        
    \begin{quote}
Il codice calcola la percentuale di valori mancanti (NaN) per ogni
colonna nel DataFrame df. Questo viene fatto utilizzando il metodo
isnull().sum() per ottenere il numero totale di valori mancanti per ogni
colonna, dividendo per la lunghezza del DataFrame len(df) per ottenere
la proporzione, e moltiplicando per 100 per convertire in percentuale.
Il risultato sarà una serie in cui l'indice è il nome della colonna e il
valore è la percentuale di valori mancanti in quella colonna.
\end{quote}

    \subsubsection{Percentuale di missing
values}\label{percentuale-di-missing-values}

    \begin{tcolorbox}[breakable, size=fbox, boxrule=1pt, pad at break*=1mm,colback=cellbackground, colframe=cellborder]
\prompt{In}{incolor}{18}{\boxspacing}
\begin{Verbatim}[commandchars=\\\{\}]
\PY{c+c1}{\PYZsh{}Calcola la percentuale di righe con missing values per ciascuna variabile}
\PY{n}{missing\PYZus{}percent}\PY{o}{=} \PY{p}{(}\PY{n}{df}\PY{o}{.}\PY{n}{isnull}\PY{p}{(}\PY{p}{)}\PY{o}{.}\PY{n}{sum}\PY{p}{(}\PY{p}{)}\PY{p}{)} \PY{o}{/} \PY{n+nb}{len}\PY{p}{(}\PY{n}{df}\PY{p}{)} \PY{o}{*} \PY{l+m+mi}{100}

\PY{c+c1}{\PYZsh{}crea il grafico a barre}
\PY{n}{plt}\PY{o}{.}\PY{n}{figure}\PY{p}{(}\PY{n}{figsize}\PY{o}{=}\PY{p}{(}\PY{l+m+mi}{10}\PY{p}{,}\PY{l+m+mi}{6}\PY{p}{)}\PY{p}{)}
\PY{n}{missing\PYZus{}percent}\PY{o}{.}\PY{n}{plot}\PY{p}{(}\PY{n}{kind}\PY{o}{=}\PY{l+s+s1}{\PYZsq{}}\PY{l+s+s1}{bar}\PY{l+s+s1}{\PYZsq{}}\PY{p}{,} \PY{n}{color}\PY{o}{=}\PY{l+s+s1}{\PYZsq{}}\PY{l+s+s1}{skyblue}\PY{l+s+s1}{\PYZsq{}}\PY{p}{,} \PY{n}{alpha}\PY{o}{=}\PY{l+m+mf}{0.8}\PY{p}{)}
\PY{n}{plt}\PY{o}{.}\PY{n}{xlabel}\PY{p}{(}\PY{l+s+s1}{\PYZsq{}}\PY{l+s+s1}{Variabile}\PY{l+s+s1}{\PYZsq{}}\PY{p}{)}
\PY{n}{plt}\PY{o}{.}\PY{n}{ylabel}\PY{p}{(}\PY{l+s+s1}{\PYZsq{}}\PY{l+s+s1}{5 di missing values}\PY{l+s+s1}{\PYZsq{}}\PY{p}{)}
\PY{n}{plt}\PY{o}{.}\PY{n}{title}\PY{p}{(}\PY{l+s+s1}{\PYZsq{}}\PY{l+s+s1}{analisi del missing values per variabile}\PY{l+s+s1}{\PYZsq{}}\PY{p}{)}
\PY{n}{plt}\PY{o}{.}\PY{n}{xticks}\PY{p}{(}\PY{n}{rotation}\PY{o}{=}\PY{l+m+mi}{0}\PY{p}{)}
\PY{n}{plt}\PY{o}{.}\PY{n}{show}\PY{p}{(}\PY{p}{)}
\end{Verbatim}
\end{tcolorbox}

    \begin{center}
    \adjustimage{max size={0.9\linewidth}{0.9\paperheight}}{output_55_0.png}
    \end{center}
    { \hspace*{\fill} \\}
    
    \begin{quote}
Il codice fa quanto segue:

Calcola la percentuale di righe con valori mancanti (NaN) per ogni
colonna nel DataFrame df. Questo viene fatto utilizzando il metodo
isnull().sum() per ottenere il numero totale di valori mancanti per ogni
colonna, dividendo per la lunghezza del DataFrame len(df) per ottenere
la proporzione, e moltiplicando per 100 per convertire in percentuale.
Il risultato è una serie in cui l'indice è il nome della colonna e il
valore è la percentuale di valori mancanti in quella colonna.

Crea una nuova figura con dimensioni specificate (10x6).

Crea un grafico a barre della serie missing\_percent utilizzando il
metodo plot con l'argomento kind=`bar'. Le barre sono colorate in
`skyblue' con un'opacità di 0.8.

Imposta le etichette degli assi x e y e il titolo del grafico.

Imposta la rotazione delle etichette dell'asse x a 0 gradi.

Infine, il comando plt.show() viene utilizzato per visualizzare la
figura.
\end{quote}

\begin{quote}
Quindi, il codice calcola la percentuale di valori mancanti per ogni
colonna nel DataFrame df e crea un grafico a barre per visualizzare
queste percentuali.
\end{quote}

    \subsubsection{Analisi Esplorativa dei Dati Iniziale del
DataFrame}\label{analisi-esplorativa-dei-dati-iniziale-del-dataframe}

    \begin{tcolorbox}[breakable, size=fbox, boxrule=1pt, pad at break*=1mm,colback=cellbackground, colframe=cellborder]
\prompt{In}{incolor}{19}{\boxspacing}
\begin{Verbatim}[commandchars=\\\{\}]
\PY{c+c1}{\PYZsh{}informazioni sul dataset }
\PY{n+nb}{print}\PY{p}{(}\PY{n}{df}\PY{o}{.}\PY{n}{info}\PY{p}{(}\PY{p}{)}\PY{p}{)}

\PY{c+c1}{\PYZsh{}statistiche desrittive}
\PY{n+nb}{print}\PY{p}{(}\PY{n}{df}\PY{o}{.}\PY{n}{describe}\PY{p}{(}\PY{p}{)}\PY{p}{)}
\end{Verbatim}
\end{tcolorbox}

    \begin{Verbatim}[commandchars=\\\{\}]
<class 'pandas.core.frame.DataFrame'>
RangeIndex: 5 entries, 0 to 4
Data columns (total 3 columns):
 \#   Column          Non-Null Count  Dtype
---  ------          --------------  -----
 0   Variable1       5 non-null      int64
 1   Variable2       3 non-null      float64
 2   Missing\_Column  4 non-null      object
dtypes: float64(1), int64(1), object(1)
memory usage: 252.0+ bytes
None
       Variable1  Variable2
count   5.000000   3.000000
mean    3.000000   2.333333
std     1.581139   1.527525
min     1.000000   1.000000
25\%     2.000000   1.500000
50\%     3.000000   2.000000
75\%     4.000000   3.000000
max     5.000000   4.000000
    \end{Verbatim}

    \begin{quote}
Il codice fa quanto segue:

Utilizza il metodo info() sul DataFrame df per stampare informazioni sul
DataFrame, come il numero di righe e colonne, i tipi di dati di ciascuna
colonna, il numero di valori non nulli in ciascuna colonna e l'uso della
memoria.

Utilizza il metodo describe() sul DataFrame df per stampare statistiche
descrittive per ciascuna colonna, come il conteggio, la media, la
deviazione standard, il minimo, il 25° percentile (Q1), la mediana (50°
percentile o Q2), il 75° percentile (Q3) e il massimo.
\end{quote}

\begin{quote}
Quindi, il codice fornisce un'analisi esplorativa dei dati (EDA)
iniziale del DataFrame df, stampando informazioni generali sul DataFrame
e statistiche descrittive per ciascuna colonna.
\end{quote}

    \subsubsection{Visualizzazione della Posizione dei Valori Mancanti nel
DataFrame tramite una Mappa
Termica}\label{visualizzazione-della-posizione-dei-valori-mancanti-nel-dataframe-tramite-una-mappa-termica}

    \begin{tcolorbox}[breakable, size=fbox, boxrule=1pt, pad at break*=1mm,colback=cellbackground, colframe=cellborder]
\prompt{In}{incolor}{20}{\boxspacing}
\begin{Verbatim}[commandchars=\\\{\}]
\PY{n}{plt}\PY{o}{.}\PY{n}{figure}\PY{p}{(}\PY{n}{figsize}\PY{o}{=}\PY{p}{(}\PY{l+m+mi}{8}\PY{p}{,}\PY{l+m+mi}{6}\PY{p}{)}\PY{p}{)}
\PY{n}{sns}\PY{o}{.}\PY{n}{heatmap}\PY{p}{(}\PY{n}{df}\PY{o}{.}\PY{n}{isnull}\PY{p}{(}\PY{p}{)}\PY{p}{,} \PY{n}{cmap}\PY{o}{=}\PY{l+s+s1}{\PYZsq{}}\PY{l+s+s1}{viridis}\PY{l+s+s1}{\PYZsq{}}\PY{p}{,} \PY{n}{cbar}\PY{o}{=}\PY{k+kc}{False}\PY{p}{,}\PY{n}{alpha}\PY{o}{=}\PY{l+m+mf}{0.8}\PY{p}{)}
\PY{n}{plt}\PY{o}{.}\PY{n}{title}\PY{p}{(}\PY{l+s+s1}{\PYZsq{}}\PY{l+s+s1}{matrice di missing values}\PY{l+s+s1}{\PYZsq{}}\PY{p}{)}
\PY{n}{plt}\PY{o}{.}\PY{n}{show}\PY{p}{(}\PY{p}{)}
\end{Verbatim}
\end{tcolorbox}

    \begin{center}
    \adjustimage{max size={0.9\linewidth}{0.9\paperheight}}{output_61_0.png}
    \end{center}
    { \hspace*{\fill} \\}
    
    \begin{quote}
Il codice crea una mappa termica (heatmap) che visualizza la posizione
dei valori mancanti nel DataFrame df. Le celle colorate nella mappa
termica rappresentano i valori mancanti. Questo è molto utile per avere
una visione rapida di dove si trovano i dati mancanti nel tuo DataFrame.
\end{quote}

    \subsubsection{Generazione e Visualizzazione di un DataFrame di Dati
Casuali}\label{generazione-e-visualizzazione-di-un-dataframe-di-dati-casuali}

    \begin{tcolorbox}[breakable, size=fbox, boxrule=1pt, pad at break*=1mm,colback=cellbackground, colframe=cellborder]
\prompt{In}{incolor}{21}{\boxspacing}
\begin{Verbatim}[commandchars=\\\{\}]
\PY{k+kn}{import} \PY{n+nn}{pandas} \PY{k}{as} \PY{n+nn}{pd}
\PY{k+kn}{import} \PY{n+nn}{numpy} \PY{k}{as} \PY{n+nn}{np}
\PY{k+kn}{import} \PY{n+nn}{matplotlib}\PY{n+nn}{.}\PY{n+nn}{pyplot} \PY{k}{as} \PY{n+nn}{plt}
\PY{k+kn}{import} \PY{n+nn}{seaborn} \PY{k}{as} \PY{n+nn}{sns}
\PY{k+kn}{import} \PY{n+nn}{plotly}\PY{n+nn}{.}\PY{n+nn}{express} \PY{k}{as} \PY{n+nn}{px}

\PY{c+c1}{\PYZsh{} Genera dati casuali per l\PYZsq{}esplorazione}
\PY{n}{np}\PY{o}{.}\PY{n}{random}\PY{o}{.}\PY{n}{seed}\PY{p}{(}\PY{l+m+mi}{42}\PY{p}{)}
\PY{n}{data} \PY{o}{=} \PY{p}{\PYZob{}}
    \PY{l+s+s1}{\PYZsq{}}\PY{l+s+s1}{Età}\PY{l+s+s1}{\PYZsq{}}\PY{p}{:} \PY{n}{np}\PY{o}{.}\PY{n}{random}\PY{o}{.}\PY{n}{randint}\PY{p}{(}\PY{l+m+mi}{18}\PY{p}{,} \PY{l+m+mi}{70}\PY{p}{,} \PY{n}{size}\PY{o}{=}\PY{l+m+mi}{1000}\PY{p}{)}\PY{p}{,}
    \PY{l+s+s1}{\PYZsq{}}\PY{l+s+s1}{Genere}\PY{l+s+s1}{\PYZsq{}}\PY{p}{:} \PY{n}{np}\PY{o}{.}\PY{n}{random}\PY{o}{.}\PY{n}{choice}\PY{p}{(}\PY{p}{[}\PY{l+s+s1}{\PYZsq{}}\PY{l+s+s1}{Maschio}\PY{l+s+s1}{\PYZsq{}}\PY{p}{,} \PY{l+s+s1}{\PYZsq{}}\PY{l+s+s1}{Femmina}\PY{l+s+s1}{\PYZsq{}}\PY{p}{]}\PY{p}{,} \PY{n}{size}\PY{o}{=}\PY{l+m+mi}{1000}\PY{p}{)}\PY{p}{,}
    \PY{l+s+s1}{\PYZsq{}}\PY{l+s+s1}{Punteggio}\PY{l+s+s1}{\PYZsq{}}\PY{p}{:} \PY{n}{np}\PY{o}{.}\PY{n}{random}\PY{o}{.}\PY{n}{uniform}\PY{p}{(}\PY{l+m+mi}{0}\PY{p}{,} \PY{l+m+mi}{100}\PY{p}{,} \PY{n}{size}\PY{o}{=}\PY{l+m+mi}{1000}\PY{p}{)}\PY{p}{,}
    \PY{l+s+s1}{\PYZsq{}}\PY{l+s+s1}{Reddito}\PY{l+s+s1}{\PYZsq{}}\PY{p}{:} \PY{n}{np}\PY{o}{.}\PY{n}{random}\PY{o}{.}\PY{n}{normal}\PY{p}{(}\PY{l+m+mi}{50000}\PY{p}{,} \PY{l+m+mi}{15000}\PY{p}{,} \PY{n}{size}\PY{o}{=}\PY{l+m+mi}{1000}\PY{p}{)} \PY{c+c1}{\PYZsh{}distribuzione gaussiana}
\PY{p}{\PYZcb{}}

\PY{n}{df} \PY{o}{=} \PY{n}{pd}\PY{o}{.}\PY{n}{DataFrame}\PY{p}{(}\PY{n}{data}\PY{p}{)}

\PY{c+c1}{\PYZsh{} Visualizza le prime righe del dataset}
\PY{n+nb}{print}\PY{p}{(}\PY{n}{df}\PY{o}{.}\PY{n}{head}\PY{p}{(}\PY{p}{)}\PY{p}{)}
\end{Verbatim}
\end{tcolorbox}

    \begin{Verbatim}[commandchars=\\\{\}]
   Età   Genere  Punteggio       Reddito
0   56  Maschio  85.120691  52915.764524
1   69  Maschio  49.514653  44702.505608
2   46  Maschio  48.058658  55077.257652
3   32  Femmina  59.240778  45568.978848
4   60  Maschio  82.468097  52526.914644
    \end{Verbatim}

    \begin{quote}
Il codice fa quanto segue:

Importa le librerie necessarie: pandas, numpy, matplotlib.pyplot,
seaborn e plotly.express.

Imposta un seed per il generatore di numeri casuali di numpy, in modo
che i risultati siano riproducibili.

Genera un dataset di 1000 righe con quattro variabili: `Età', `Genere',
`Punteggio' e `Reddito'. I dati sono generati casualmente utilizzando
diverse distribuzioni: uniforme, normale e una scelta casuale tra due
opzioni.

Crea un DataFrame pandas df da questo dataset.

Infine, stampa le prime cinque righe del DataFrame utilizzando il metodo
head().
\end{quote}

\begin{quote}
Quindi, il codice genera un DataFrame di dati casuali e stampa le prime
cinque righe.
\end{quote}

    \subsubsection{Calcolo e Stampa del Numero Totale di Valori Mancanti per
Colonna nel
DataFrame}\label{calcolo-e-stampa-del-numero-totale-di-valori-mancanti-per-colonna-nel-dataframe}

    \begin{tcolorbox}[breakable, size=fbox, boxrule=1pt, pad at break*=1mm,colback=cellbackground, colframe=cellborder]
\prompt{In}{incolor}{22}{\boxspacing}
\begin{Verbatim}[commandchars=\\\{\}]
\PY{c+c1}{\PYZsh{}gesione valori mancanti}
\PY{n}{missing\PYZus{}data} \PY{o}{=} \PY{n}{df}\PY{o}{.}\PY{n}{isnull}\PY{p}{(}\PY{p}{)}\PY{o}{.}\PY{n}{sum}\PY{p}{(}\PY{p}{)}
\PY{n+nb}{print}\PY{p}{(}\PY{l+s+s1}{\PYZsq{}}\PY{l+s+s1}{valori mancanti per ciascuna colonna: }\PY{l+s+s1}{\PYZsq{}}\PY{p}{)}
\PY{n+nb}{print}\PY{p}{(}\PY{n}{missing\PYZus{}data}\PY{p}{)}
\end{Verbatim}
\end{tcolorbox}

    \begin{Verbatim}[commandchars=\\\{\}]
valori mancanti per ciascuna colonna:
Età          0
Genere       0
Punteggio    0
Reddito      0
dtype: int64
    \end{Verbatim}

    \begin{quote}
Il codice fa quanto segue:

Utilizza il metodo isnull().sum() sul DataFrame df per calcolare il
numero totale di valori mancanti (NaN) per ogni colonna. Il risultato è
una serie in cui l'indice è il nome della colonna e il valore è il
numero totale di valori mancanti in quella colonna.

Infine, stampa la frase `valori mancanti per ciascuna colonna:' e la
serie missing\_data.
\end{quote}

\begin{quote}
Quindi, il codice calcola e stampa il numero totale di valori mancanti
per ogni colonna nel DataFrame df.
\end{quote}

    \subsubsection{Creazione e Visualizzazione di un Istogramma della
Distribuzione dei Valori
Numerici}\label{creazione-e-visualizzazione-di-un-istogramma-della-distribuzione-dei-valori-numerici}

    \begin{tcolorbox}[breakable, size=fbox, boxrule=1pt, pad at break*=1mm,colback=cellbackground, colframe=cellborder]
\prompt{In}{incolor}{23}{\boxspacing}
\begin{Verbatim}[commandchars=\\\{\}]
\PY{c+c1}{\PYZsh{}visualizza la distribuzione dele varibaili numeriche}

\PY{n}{plt}\PY{o}{.}\PY{n}{figure}\PY{p}{(}\PY{n}{figsize}\PY{o}{=}\PY{p}{(}\PY{l+m+mi}{12}\PY{p}{,}\PY{l+m+mi}{6}\PY{p}{)}\PY{p}{)}
\PY{n}{sns}\PY{o}{.}\PY{n}{set\PYZus{}style}\PY{p}{(}\PY{l+s+s1}{\PYZsq{}}\PY{l+s+s1}{whitegrid}\PY{l+s+s1}{\PYZsq{}}\PY{p}{)}
\PY{n}{sns}\PY{o}{.}\PY{n}{histplot}\PY{p}{(}\PY{n}{df}\PY{p}{[}\PY{l+s+s1}{\PYZsq{}}\PY{l+s+s1}{Punteggio}\PY{l+s+s1}{\PYZsq{}}\PY{p}{]}\PY{p}{,} \PY{n}{kde}\PY{o}{=}\PY{k+kc}{False}\PY{p}{,} \PY{n}{bins}\PY{o}{=}\PY{l+m+mi}{50}\PY{p}{,} \PY{n}{label}\PY{o}{=}\PY{l+s+s1}{\PYZsq{}}\PY{l+s+s1}{Punteggio}\PY{l+s+s1}{\PYZsq{}}\PY{p}{)} \PY{c+c1}{\PYZsh{}istogramma}
\PY{n}{plt}\PY{o}{.}\PY{n}{legend}\PY{p}{(}\PY{p}{)}
\PY{n}{plt}\PY{o}{.}\PY{n}{title}\PY{p}{(}\PY{l+s+s1}{\PYZsq{}}\PY{l+s+s1}{distribuzione delle variabili nuemriche}\PY{l+s+s1}{\PYZsq{}}\PY{p}{)}
\PY{n}{plt}\PY{o}{.}\PY{n}{show}\PY{p}{(}\PY{p}{)}
\end{Verbatim}
\end{tcolorbox}

    \begin{center}
    \adjustimage{max size={0.9\linewidth}{0.9\paperheight}}{output_70_0.png}
    \end{center}
    { \hspace*{\fill} \\}
    
    \begin{quote}
Il codice fa quanto segue:

Crea una nuova figura con dimensioni specificate (12x6).

Imposta lo stile del grafico a `whitegrid' utilizzando sns.set\_style.

Crea un istogramma della colonna `Punteggio' del DataFrame dfutilizzando
sns.histplot. L'argomento kde=False indica che non vuoi visualizzare la
curva di densità del kernel, e bins=50 indica che vuoi dividere i dati
in 50 barre (o intervalli). L'argomento label=`Punteggio' fornisce
un'etichetta per i dati.

Visualizza la legenda del grafico con plt.legend().

Imposta il titolo del grafico come `distribuzione delle variabili
numeriche'.

Infine, il comando plt.show() viene utilizzato per visualizzare la
figura.
\end{quote}

\begin{quote}
Quindi, il codice crea e visualizza un istogramma della distribuzione
dei valori nella colonna `Punteggio' del DataFrame df.
\end{quote}

    \subsubsection{Creazione e Visualizzazione di un Box Plot della
Distribuzione dei Punteggi per
Genere}\label{creazione-e-visualizzazione-di-un-box-plot-della-distribuzione-dei-punteggi-per-genere}

    \begin{tcolorbox}[breakable, size=fbox, boxrule=1pt, pad at break*=1mm,colback=cellbackground, colframe=cellborder]
\prompt{In}{incolor}{24}{\boxspacing}
\begin{Verbatim}[commandchars=\\\{\}]
\PY{c+c1}{\PYZsh{}visualizza una box plot per una variabile numerica rispetto a un}
\PY{n}{plt}\PY{o}{.}\PY{n}{figure}\PY{p}{(}\PY{n}{figsize}\PY{o}{=}\PY{p}{(}\PY{l+m+mi}{10}\PY{p}{,}\PY{l+m+mi}{6}\PY{p}{)}\PY{p}{)}
\PY{n}{sns}\PY{o}{.}\PY{n}{boxplot}\PY{p}{(}\PY{n}{x}\PY{o}{=}\PY{l+s+s1}{\PYZsq{}}\PY{l+s+s1}{Genere}\PY{l+s+s1}{\PYZsq{}}\PY{p}{,}\PY{n}{y}\PY{o}{=}\PY{l+s+s1}{\PYZsq{}}\PY{l+s+s1}{Punteggio}\PY{l+s+s1}{\PYZsq{}}\PY{p}{,} \PY{n}{data}\PY{o}{=}\PY{n}{df}\PY{p}{)}
\PY{n}{plt}\PY{o}{.}\PY{n}{title}\PY{p}{(}\PY{l+s+s1}{\PYZsq{}}\PY{l+s+s1}{box plot tra genere e punteggio}\PY{l+s+s1}{\PYZsq{}}\PY{p}{)}
\PY{n}{plt}\PY{o}{.}\PY{n}{show}\PY{p}{(}\PY{p}{)}
\end{Verbatim}
\end{tcolorbox}

    \begin{center}
    \adjustimage{max size={0.9\linewidth}{0.9\paperheight}}{output_73_0.png}
    \end{center}
    { \hspace*{\fill} \\}
    
    \begin{quote}
Il codice fa quanto segue:

Crea una nuova figura con dimensioni specificate (10x6).

Utilizza la funzione boxplot della libreria seaborn (alias sns) per
creare un box plot della colonna `Punteggio' del DataFrame df,
raggruppato per la colonna `Genere'. Questo grafico mostra la
distribuzione dei punteggi per ciascun genere.

Imposta il titolo del grafico come `box plot tra genere e punteggio'.

Infine, il comando plt.show() viene utilizzato per visualizzare la
figura.
\end{quote}

\begin{quote}
Quindi, il codice crea e visualizza un box plot che mostra la
distribuzione dei punteggi per ciascun genere nel DataFrame df.
\end{quote}

    \subsubsection{Creazione e Visualizzazione di un Grafico a Dispersione
Interattivo con
Plotly}\label{creazione-e-visualizzazione-di-un-grafico-a-dispersione-interattivo-con-plotly}

    \begin{tcolorbox}[breakable, size=fbox, boxrule=1pt, pad at break*=1mm,colback=cellbackground, colframe=cellborder]
\prompt{In}{incolor}{25}{\boxspacing}
\begin{Verbatim}[commandchars=\\\{\}]
\PY{c+c1}{\PYZsh{}visualizza un grafico a dispersione interattivo utilizzando platly}
\PY{k+kn}{import} \PY{n+nn}{plotly}\PY{n+nn}{.}\PY{n+nn}{express} \PY{k}{as} \PY{n+nn}{px}
\PY{n}{fig} \PY{o}{=} \PY{n}{px}\PY{o}{.}\PY{n}{scatter}\PY{p}{(}\PY{n}{df}\PY{p}{,} \PY{n}{x}\PY{o}{=}\PY{l+s+s1}{\PYZsq{}}\PY{l+s+s1}{Età}\PY{l+s+s1}{\PYZsq{}}\PY{p}{,} \PY{n}{y}\PY{o}{=}\PY{l+s+s1}{\PYZsq{}}\PY{l+s+s1}{Reddito}\PY{l+s+s1}{\PYZsq{}}\PY{p}{,} \PY{n}{color}\PY{o}{=}\PY{l+s+s1}{\PYZsq{}}\PY{l+s+s1}{Genere}\PY{l+s+s1}{\PYZsq{}}\PY{p}{,} \PY{n}{size}\PY{o}{=}\PY{l+s+s1}{\PYZsq{}}\PY{l+s+s1}{Punteggio}\PY{l+s+s1}{\PYZsq{}}\PY{p}{)}
\PY{n}{fig}\PY{o}{.}\PY{n}{update\PYZus{}layout}\PY{p}{(}\PY{n}{title}\PY{o}{=}\PY{l+s+s1}{\PYZsq{}}\PY{l+s+s1}{Grafico a dispersione interattivo}\PY{l+s+s1}{\PYZsq{}}\PY{p}{)}
\PY{n}{fig}\PY{o}{.}\PY{n}{show}\PY{p}{(}\PY{p}{)}
\end{Verbatim}
\end{tcolorbox}

    
    
    
    
    \begin{quote}
Il codice fa quanto segue:

Importa la libreria plotly.express con l'alias px.

Crea un grafico a dispersione interattivo utilizzando la funzione
scatter di plotly.express. Le coordinate x e y del grafico sono
rispettivamente le colonne `Età' e `Reddito' del DataFrame df. I punti
sono colorati in base alla colonna `Genere' e la loro dimensione è
determinata dalla colonna `Punteggio'.

Aggiorna il layout del grafico per impostare il titolo come `Grafico a
dispersione interattivo'.

Infine, il comando fig.show() viene utilizzato per visualizzare il
grafico.
\end{quote}

\begin{quote}
Quindi, il codice crea e visualizza un grafico a dispersione interattivo
che mostra la relazione tra `Età' e `Reddito', con i punti colorati in
base al `Genere' e dimensionati in base al `Punteggio'.
\end{quote}

    \subsubsection{Generazione di un DataFrame di Dati Casuali con Valori
Mancanti}\label{generazione-di-un-dataframe-di-dati-casuali-con-valori-mancanti}

    \begin{tcolorbox}[breakable, size=fbox, boxrule=1pt, pad at break*=1mm,colback=cellbackground, colframe=cellborder]
\prompt{In}{incolor}{26}{\boxspacing}
\begin{Verbatim}[commandchars=\\\{\}]
\PY{k+kn}{import} \PY{n+nn}{pandas} \PY{k}{as} \PY{n+nn}{pd}
\PY{k+kn}{import} \PY{n+nn}{numpy} \PY{k}{as} \PY{n+nn}{np}

\PY{c+c1}{\PYZsh{} Impostare il seed per rendere i risultati riproducibili}
\PY{n}{np}\PY{o}{.}\PY{n}{random}\PY{o}{.}\PY{n}{seed}\PY{p}{(}\PY{l+m+mi}{41}\PY{p}{)}

\PY{c+c1}{\PYZsh{} Creare un dataframe vuoto}
\PY{n}{df} \PY{o}{=} \PY{n}{pd}\PY{o}{.}\PY{n}{DataFrame}\PY{p}{(}\PY{p}{)}

\PY{c+c1}{\PYZsh{} Generare dati casuali}
\PY{n}{n\PYZus{}rows} \PY{o}{=} \PY{l+m+mi}{10000}
\PY{n}{df}\PY{p}{[}\PY{l+s+s1}{\PYZsq{}}\PY{l+s+s1}{CatCol1}\PY{l+s+s1}{\PYZsq{}}\PY{p}{]} \PY{o}{=} \PY{n}{np}\PY{o}{.}\PY{n}{random}\PY{o}{.}\PY{n}{choice}\PY{p}{(}\PY{p}{[}\PY{l+s+s1}{\PYZsq{}}\PY{l+s+s1}{A}\PY{l+s+s1}{\PYZsq{}}\PY{p}{,} \PY{l+s+s1}{\PYZsq{}}\PY{l+s+s1}{B}\PY{l+s+s1}{\PYZsq{}}\PY{p}{,} \PY{l+s+s1}{\PYZsq{}}\PY{l+s+s1}{C}\PY{l+s+s1}{\PYZsq{}}\PY{p}{]}\PY{p}{,} \PY{n}{size}\PY{o}{=}\PY{n}{n\PYZus{}rows}\PY{p}{)}
\PY{n}{df}\PY{p}{[}\PY{l+s+s1}{\PYZsq{}}\PY{l+s+s1}{CatCol2}\PY{l+s+s1}{\PYZsq{}}\PY{p}{]} \PY{o}{=} \PY{n}{np}\PY{o}{.}\PY{n}{random}\PY{o}{.}\PY{n}{choice}\PY{p}{(}\PY{p}{[}\PY{l+s+s1}{\PYZsq{}}\PY{l+s+s1}{X}\PY{l+s+s1}{\PYZsq{}}\PY{p}{,} \PY{l+s+s1}{\PYZsq{}}\PY{l+s+s1}{Y}\PY{l+s+s1}{\PYZsq{}}\PY{p}{]}\PY{p}{,} \PY{n}{size}\PY{o}{=}\PY{n}{n\PYZus{}rows}\PY{p}{)}
\PY{n}{df}\PY{p}{[}\PY{l+s+s1}{\PYZsq{}}\PY{l+s+s1}{NumCol1}\PY{l+s+s1}{\PYZsq{}}\PY{p}{]} \PY{o}{=} \PY{n}{np}\PY{o}{.}\PY{n}{random}\PY{o}{.}\PY{n}{randn}\PY{p}{(}\PY{n}{n\PYZus{}rows}\PY{p}{)}
\PY{n}{df}\PY{p}{[}\PY{l+s+s1}{\PYZsq{}}\PY{l+s+s1}{NumCol2}\PY{l+s+s1}{\PYZsq{}}\PY{p}{]} \PY{o}{=} \PY{n}{np}\PY{o}{.}\PY{n}{random}\PY{o}{.}\PY{n}{randint}\PY{p}{(}\PY{l+m+mi}{1}\PY{p}{,} \PY{l+m+mi}{100}\PY{p}{,} \PY{n}{size}\PY{o}{=}\PY{n}{n\PYZus{}rows}\PY{p}{)}
\PY{n}{df}\PY{p}{[}\PY{l+s+s1}{\PYZsq{}}\PY{l+s+s1}{NumCol3}\PY{l+s+s1}{\PYZsq{}}\PY{p}{]} \PY{o}{=} \PY{n}{np}\PY{o}{.}\PY{n}{random}\PY{o}{.}\PY{n}{uniform}\PY{p}{(}\PY{l+m+mi}{0}\PY{p}{,} \PY{l+m+mi}{1}\PY{p}{,} \PY{n}{size}\PY{o}{=}\PY{n}{n\PYZus{}rows}\PY{p}{)}

\PY{c+c1}{\PYZsh{} Calcolare il numero totale di missing values desiderati}
\PY{n}{total\PYZus{}missing\PYZus{}values} \PY{o}{=} \PY{n+nb}{int}\PY{p}{(}\PY{l+m+mf}{0.03} \PY{o}{*} \PY{n}{n\PYZus{}rows} \PY{o}{*} \PY{n+nb}{len}\PY{p}{(}\PY{n}{df}\PY{o}{.}\PY{n}{columns}\PY{p}{)}\PY{p}{)}

\PY{c+c1}{\PYZsh{} Introdurre missing values casuali}
\PY{k}{for} \PY{n}{column} \PY{o+ow}{in} \PY{n}{df}\PY{o}{.}\PY{n}{columns}\PY{p}{:}
    \PY{n}{num\PYZus{}missing\PYZus{}values} \PY{o}{=} \PY{n}{np}\PY{o}{.}\PY{n}{random}\PY{o}{.}\PY{n}{randint}\PY{p}{(}\PY{l+m+mi}{0}\PY{p}{,} \PY{n}{total\PYZus{}missing\PYZus{}values} \PY{o}{+} \PY{l+m+mi}{1}\PY{p}{)}
    \PY{n}{missing\PYZus{}indices} \PY{o}{=} \PY{n}{np}\PY{o}{.}\PY{n}{random}\PY{o}{.}\PY{n}{choice}\PY{p}{(}\PY{n}{n\PYZus{}rows}\PY{p}{,} \PY{n}{size}\PY{o}{=}\PY{n}{num\PYZus{}missing\PYZus{}values}\PY{p}{,} \PY{n}{replace}\PY{o}{=}\PY{k+kc}{False}\PY{p}{)}
    \PY{n}{df}\PY{o}{.}\PY{n}{loc}\PY{p}{[}\PY{n}{missing\PYZus{}indices}\PY{p}{,} \PY{n}{column}\PY{p}{]} \PY{o}{=} \PY{n}{np}\PY{o}{.}\PY{n}{nan}
\PY{n}{df}
\end{Verbatim}
\end{tcolorbox}

            \begin{tcolorbox}[breakable, size=fbox, boxrule=.5pt, pad at break*=1mm, opacityfill=0]
\prompt{Out}{outcolor}{26}{\boxspacing}
\begin{Verbatim}[commandchars=\\\{\}]
     CatCol1 CatCol2   NumCol1  NumCol2   NumCol3
0          A     NaN  0.440877     49.0  0.246007
1          A       Y  1.945879     28.0  0.936825
2          C       X  0.988834     42.0  0.751516
3          A       Y -0.181978     73.0  0.950696
4          B       X  2.080615     74.0  0.903045
{\ldots}      {\ldots}     {\ldots}       {\ldots}      {\ldots}       {\ldots}
9995       C       Y  1.352114     61.0  0.728445
9996       C       Y  1.143642     67.0  0.605930
9997       A       X -0.665794     54.0  0.071041
9998       C       Y  0.004278      NaN       NaN
9999       A       X  0.622473     95.0  0.751384

[10000 rows x 5 columns]
\end{Verbatim}
\end{tcolorbox}
        
    \begin{quote}
Il codice fa quanto segue:

Importa le librerie pandas e numpy.

Imposta un seed per il generatore di numeri casuali di numpy, in modo
che i risultati siano riproducibili.

Crea un DataFrame vuoto df.

Genera dati casuali per cinque colonne nel DataFrame: `CatCol1',
`CatCol2', `NumCol1', `NumCol2' e `NumCol3'. I dati sono generati
utilizzando diverse distribuzioni: scelta casuale tra opzioni
specificate, distribuzione normale, distribuzione uniforme discreta e
distribuzione uniforme continua.

Calcola il numero totale di valori mancanti desiderati come il 3\% del
numero totale di valori nel DataFrame.

Introduce valori mancanti (NaN) casuali nel DataFrame. Per ogni colonna,
sceglie un numero casuale di righe e imposta il valore in quelle righe a
NaN.

Infine, stampa il DataFrame df.
\end{quote}

\begin{quote}
Quindi, il codice genera un DataFrame di dati casuali e introduce valori
mancanti casuali.
\end{quote}

    \subsubsection{Creazione e Visualizzazione di una Mappa Termica dei
Valori Mancanti nel
DataFrame}\label{creazione-e-visualizzazione-di-una-mappa-termica-dei-valori-mancanti-nel-dataframe}

    \begin{tcolorbox}[breakable, size=fbox, boxrule=1pt, pad at break*=1mm,colback=cellbackground, colframe=cellborder]
\prompt{In}{incolor}{27}{\boxspacing}
\begin{Verbatim}[commandchars=\\\{\}]
\PY{n}{missing\PYZus{}matrix} \PY{o}{=} \PY{n}{df}\PY{o}{.}\PY{n}{isnull}\PY{p}{(}\PY{p}{)}
\PY{c+c1}{\PYZsh{}crea una heatmap colorata}
\PY{n}{plt}\PY{o}{.}\PY{n}{figure}\PY{p}{(}\PY{n}{figsize}\PY{o}{=}\PY{p}{(}\PY{l+m+mi}{8}\PY{p}{,}\PY{l+m+mi}{6}\PY{p}{)}\PY{p}{)}
\PY{n}{sns}\PY{o}{.}\PY{n}{heatmap}\PY{p}{(}\PY{n}{missing\PYZus{}matrix}\PY{p}{,} \PY{n}{cmap}\PY{o}{=}\PY{l+s+s1}{\PYZsq{}}\PY{l+s+s1}{viridis}\PY{l+s+s1}{\PYZsq{}}\PY{p}{,} \PY{n}{cbar}\PY{o}{=}\PY{k+kc}{False}\PY{p}{,}\PY{n}{alpha}\PY{o}{=}\PY{l+m+mf}{0.8}\PY{p}{)}
\PY{n}{plt}\PY{o}{.}\PY{n}{title}\PY{p}{(}\PY{l+s+s1}{\PYZsq{}}\PY{l+s+s1}{Matrice di missing values}\PY{l+s+s1}{\PYZsq{}}\PY{p}{)}
\PY{n}{plt}\PY{o}{.}\PY{n}{show}\PY{p}{(}\PY{p}{)}
\end{Verbatim}
\end{tcolorbox}

    \begin{center}
    \adjustimage{max size={0.9\linewidth}{0.9\paperheight}}{output_82_0.png}
    \end{center}
    { \hspace*{\fill} \\}
    
    \begin{quote}
Il codice fa quanto segue:

Crea una matrice booleana missing\_matrix che indica la posizione dei
valori mancanti (NaN) nel DataFrame df utilizzando il metodo isnull().

Crea una nuova figura con dimensioni specificate (8x6).

Utilizza la funzione heatmap della libreria seaborn (alias sns) per
creare una mappa termica (heatmap) della matrice missing\_matrix. La
mappa termica è colorata usando il colormap `viridis'. L'argomento
cbar=False indica che non vuoi visualizzare una barra di colore che
mostra la scala dei colori utilizzati nella mappa termica. L'argomento
alpha=0.8 imposta l'opacità dei colori nella mappa termica.

Imposta il titolo della mappa termica come `Matrice di missing values'.

Infine, il comando plt.show() viene utilizzato per visualizzare la
figura.
\end{quote}

\begin{quote}
Quindi, il codice crea e visualizza una mappa termica dei valori
mancanti nel DataFrame df.
\end{quote}

    \subsubsection{Generazione e Visualizzazione di un DataFrame di Dati
Casuali}\label{generazione-e-visualizzazione-di-un-dataframe-di-dati-casuali}

    \begin{tcolorbox}[breakable, size=fbox, boxrule=1pt, pad at break*=1mm,colback=cellbackground, colframe=cellborder]
\prompt{In}{incolor}{28}{\boxspacing}
\begin{Verbatim}[commandchars=\\\{\}]
\PY{k+kn}{import} \PY{n+nn}{pandas} \PY{k}{as} \PY{n+nn}{pd}
\PY{k+kn}{import} \PY{n+nn}{numpy} \PY{k}{as} \PY{n+nn}{np}
\PY{k+kn}{import} \PY{n+nn}{matplotlib}\PY{n+nn}{.}\PY{n+nn}{pyplot} \PY{k}{as} \PY{n+nn}{plt}
\PY{k+kn}{import} \PY{n+nn}{seaborn} \PY{k}{as} \PY{n+nn}{sns}

\PY{c+c1}{\PYZsh{} Genera dati casuali per l\PYZsq{}esplorazione}
\PY{n}{np}\PY{o}{.}\PY{n}{random}\PY{o}{.}\PY{n}{seed}\PY{p}{(}\PY{l+m+mi}{42}\PY{p}{)}
\PY{n}{data} \PY{o}{=} \PY{p}{\PYZob{}}
    \PY{l+s+s1}{\PYZsq{}}\PY{l+s+s1}{Data}\PY{l+s+s1}{\PYZsq{}}\PY{p}{:} \PY{n}{pd}\PY{o}{.}\PY{n}{date\PYZus{}range}\PY{p}{(}\PY{n}{start}\PY{o}{=}\PY{l+s+s1}{\PYZsq{}}\PY{l+s+s1}{2023\PYZhy{}01\PYZhy{}01}\PY{l+s+s1}{\PYZsq{}}\PY{p}{,} \PY{n}{end}\PY{o}{=}\PY{l+s+s1}{\PYZsq{}}\PY{l+s+s1}{2023\PYZhy{}12\PYZhy{}31}\PY{l+s+s1}{\PYZsq{}}\PY{p}{,} \PY{n}{freq}\PY{o}{=}\PY{l+s+s1}{\PYZsq{}}\PY{l+s+s1}{D}\PY{l+s+s1}{\PYZsq{}}\PY{p}{)}\PY{p}{,} \PY{c+c1}{\PYZsh{}data casuale}
    \PY{l+s+s1}{\PYZsq{}}\PY{l+s+s1}{Vendite}\PY{l+s+s1}{\PYZsq{}}\PY{p}{:} \PY{n}{np}\PY{o}{.}\PY{n}{random}\PY{o}{.}\PY{n}{randint}\PY{p}{(}\PY{l+m+mi}{100}\PY{p}{,} \PY{l+m+mi}{1000}\PY{p}{,} \PY{n}{size}\PY{o}{=}\PY{l+m+mi}{365}\PY{p}{)}\PY{p}{,} \PY{c+c1}{\PYZsh{}100 numeri casuali tra1 100 e 1000}
    \PY{l+s+s1}{\PYZsq{}}\PY{l+s+s1}{Prodotto}\PY{l+s+s1}{\PYZsq{}}\PY{p}{:} \PY{n}{np}\PY{o}{.}\PY{n}{random}\PY{o}{.}\PY{n}{choice}\PY{p}{(}\PY{p}{[}\PY{l+s+s1}{\PYZsq{}}\PY{l+s+s1}{A}\PY{l+s+s1}{\PYZsq{}}\PY{p}{,} \PY{l+s+s1}{\PYZsq{}}\PY{l+s+s1}{B}\PY{l+s+s1}{\PYZsq{}}\PY{p}{,} \PY{l+s+s1}{\PYZsq{}}\PY{l+s+s1}{C}\PY{l+s+s1}{\PYZsq{}}\PY{p}{]}\PY{p}{,} \PY{n}{size}\PY{o}{=}\PY{l+m+mi}{365}\PY{p}{)} \PY{c+c1}{\PYZsh{}}
\PY{p}{\PYZcb{}}

\PY{n}{df} \PY{o}{=} \PY{n}{pd}\PY{o}{.}\PY{n}{DataFrame}\PY{p}{(}\PY{n}{data}\PY{p}{)}

\PY{c+c1}{\PYZsh{} Visualizza le prime righe del dataset}
\PY{n+nb}{print}\PY{p}{(}\PY{n}{df}\PY{o}{.}\PY{n}{head}\PY{p}{(}\PY{p}{)}\PY{p}{)}
\end{Verbatim}
\end{tcolorbox}

    \begin{Verbatim}[commandchars=\\\{\}]
        Data  Vendite Prodotto
0 2023-01-01      202        B
1 2023-01-02      535        A
2 2023-01-03      960        C
3 2023-01-04      370        A
4 2023-01-05      206        A
    \end{Verbatim}

    \begin{quote}
Il tuo codice fa quanto segue:

Importa le librerie necessarie: panda, numpy, matplotlib.pyplot e
seaborn.

Imposta un seed per il generatore di numeri casuali di numpy, in modo
che i risultati siano riproducibili.

Genera un dataset di 365 righe con tre variabili: `Data', `Vendite' e
`Prodotto'. I dati sono generati utilizzando diverse funzioni:
pd.date\_range genera una serie di date, np.random.randint genera numeri
interi casuali e np.random.choice sceglie casualmente tra le opzioni
specificate.

Crea un DataFrame pandas df da questo dataset.

Infine, stampa le prime cinque righe del DataFrame utilizzando il metodo
head().
\end{quote}

\begin{quote}
Quindi, il tuo codice genera un DataFrame di dati casuali e stampa le
prime cinque righe.
\end{quote}

    \subsubsection{Creazione e Visualizzazione di un Grafico delle Vendite
nel Tempo e un Box Plot delle Vendite per
Prodotto}\label{creazione-e-visualizzazione-di-un-grafico-delle-vendite-nel-tempo-e-un-box-plot-delle-vendite-per-prodotto}

    \begin{tcolorbox}[breakable, size=fbox, boxrule=1pt, pad at break*=1mm,colback=cellbackground, colframe=cellborder]
\prompt{In}{incolor}{29}{\boxspacing}
\begin{Verbatim}[commandchars=\\\{\}]
\PY{c+c1}{\PYZsh{} Visualizza un grafico delle vendite nel tempo}
\PY{n}{plt}\PY{o}{.}\PY{n}{figure}\PY{p}{(}\PY{n}{figsize}\PY{o}{=}\PY{p}{(}\PY{l+m+mi}{12}\PY{p}{,} \PY{l+m+mi}{6}\PY{p}{)}\PY{p}{)}
\PY{n}{sns}\PY{o}{.}\PY{n}{lineplot}\PY{p}{(}\PY{n}{x}\PY{o}{=}\PY{l+s+s1}{\PYZsq{}}\PY{l+s+s1}{Data}\PY{l+s+s1}{\PYZsq{}}\PY{p}{,} \PY{n}{y}\PY{o}{=}\PY{l+s+s1}{\PYZsq{}}\PY{l+s+s1}{Vendite}\PY{l+s+s1}{\PYZsq{}}\PY{p}{,} \PY{n}{data}\PY{o}{=}\PY{n}{df}\PY{p}{)}
\PY{n}{plt}\PY{o}{.}\PY{n}{title}\PY{p}{(}\PY{l+s+s1}{\PYZsq{}}\PY{l+s+s1}{Andamento delle vendite nel tempo}\PY{l+s+s1}{\PYZsq{}}\PY{p}{)}
\PY{n}{plt}\PY{o}{.}\PY{n}{xlabel}\PY{p}{(}\PY{l+s+s1}{\PYZsq{}}\PY{l+s+s1}{Data}\PY{l+s+s1}{\PYZsq{}}\PY{p}{)}
\PY{n}{plt}\PY{o}{.}\PY{n}{ylabel}\PY{p}{(}\PY{l+s+s1}{\PYZsq{}}\PY{l+s+s1}{Vendite}\PY{l+s+s1}{\PYZsq{}}\PY{p}{)}
\PY{n}{plt}\PY{o}{.}\PY{n}{xticks}\PY{p}{(}\PY{n}{rotation}\PY{o}{=}\PY{l+m+mi}{45}\PY{p}{)}
\PY{n}{plt}\PY{o}{.}\PY{n}{show}\PY{p}{(}\PY{p}{)}

\PY{c+c1}{\PYZsh{} Visualizza una box plot delle vendite per prodotto}
\PY{n}{plt}\PY{o}{.}\PY{n}{figure}\PY{p}{(}\PY{n}{figsize}\PY{o}{=}\PY{p}{(}\PY{l+m+mi}{10}\PY{p}{,} \PY{l+m+mi}{6}\PY{p}{)}\PY{p}{)}
\PY{n}{sns}\PY{o}{.}\PY{n}{boxplot}\PY{p}{(}\PY{n}{x}\PY{o}{=}\PY{l+s+s1}{\PYZsq{}}\PY{l+s+s1}{Prodotto}\PY{l+s+s1}{\PYZsq{}}\PY{p}{,} \PY{n}{y}\PY{o}{=}\PY{l+s+s1}{\PYZsq{}}\PY{l+s+s1}{Vendite}\PY{l+s+s1}{\PYZsq{}}\PY{p}{,} \PY{n}{data}\PY{o}{=}\PY{n}{df}\PY{p}{)}
\PY{n}{plt}\PY{o}{.}\PY{n}{title}\PY{p}{(}\PY{l+s+s1}{\PYZsq{}}\PY{l+s+s1}{Box Plot delle vendite per prodotto}\PY{l+s+s1}{\PYZsq{}}\PY{p}{)}
\PY{n}{plt}\PY{o}{.}\PY{n}{xlabel}\PY{p}{(}\PY{l+s+s1}{\PYZsq{}}\PY{l+s+s1}{Prodotto}\PY{l+s+s1}{\PYZsq{}}\PY{p}{)}
\PY{n}{plt}\PY{o}{.}\PY{n}{ylabel}\PY{p}{(}\PY{l+s+s1}{\PYZsq{}}\PY{l+s+s1}{Vendite}\PY{l+s+s1}{\PYZsq{}}\PY{p}{)}
\PY{n}{plt}\PY{o}{.}\PY{n}{show}\PY{p}{(}\PY{p}{)}
\end{Verbatim}
\end{tcolorbox}

    \begin{center}
    \adjustimage{max size={0.9\linewidth}{0.9\paperheight}}{output_88_0.png}
    \end{center}
    { \hspace*{\fill} \\}
    
    \begin{center}
    \adjustimage{max size={0.9\linewidth}{0.9\paperheight}}{output_88_1.png}
    \end{center}
    { \hspace*{\fill} \\}
    
    \begin{quote}
Il codice fa quanto segue:
\end{quote}

Crea un grafico delle vendite nel tempo:

Crea una nuova figura con dimensioni specificate (12x6).

Utilizza la funzione lineplot della libreria seaborn (alias sns) per
creare un grafico delle vendite nel tempo. L'asse x rappresenta la data
e l'asse y rappresenta le vendite.

Imposta il titolo del grafico come `Andamento delle vendite nel tempo' e
le etichette degli assi x e y come `Data' e `Vendite' rispettivamente.

Ruota le etichette dell'asse x di 45 gradi per una migliore leggibilità.

Infine, il comando plt.show() viene utilizzato per visualizzare il
grafico.

Crea un box plot delle vendite per prodotto:

Crea una nuova figura con dimensioni specificate (10x6).

Utilizza la funzione boxplot della libreria seaborn per creare un box
plot delle vendite per prodotto. L'asse x rappresenta il prodotto e
l'asse y rappresenta le vendite.

Imposta il titolo del grafico come `Box Plot delle vendite per prodotto'
e le etichette degli assi x e y come `Prodotto' e `Vendite'
rispettivamente.

Infine, il comando plt.show() viene utilizzato per visualizzare il
grafico.

\begin{verbatim}
</ol>
\end{verbatim}

\begin{quote}
Quindi, il codice crea e visualizza due grafici: un grafico delle
vendite nel tempo e un box plot delle vendite per prodotto.
\end{quote}

    \subsubsection{Calcolo della Media Condizionata dell'Età per Livello di
Soddisfazione e Creazione di un Grafico a
Barre}\label{calcolo-della-media-condizionata-delletuxe0-per-livello-di-soddisfazione-e-creazione-di-un-grafico-a-barre}

    \begin{tcolorbox}[breakable, size=fbox, boxrule=1pt, pad at break*=1mm,colback=cellbackground, colframe=cellborder]
\prompt{In}{incolor}{30}{\boxspacing}
\begin{Verbatim}[commandchars=\\\{\}]
\PY{k+kn}{import} \PY{n+nn}{pandas} \PY{k}{as} \PY{n+nn}{pd}
\PY{k+kn}{import} \PY{n+nn}{numpy} \PY{k}{as} \PY{n+nn}{np}
\PY{k+kn}{import} \PY{n+nn}{matplotlib}\PY{n+nn}{.}\PY{n+nn}{pyplot} \PY{k}{as} \PY{n+nn}{plt}
\PY{k+kn}{import} \PY{n+nn}{seaborn} \PY{k}{as} \PY{n+nn}{sns}

\PY{c+c1}{\PYZsh{} Genera dati casuali per l\PYZsq{}esplorazione}
\PY{n}{np}\PY{o}{.}\PY{n}{random}\PY{o}{.}\PY{n}{seed}\PY{p}{(}\PY{l+m+mi}{42}\PY{p}{)}
\PY{n}{data} \PY{o}{=} \PY{p}{\PYZob{}}
    \PY{l+s+s1}{\PYZsq{}}\PY{l+s+s1}{Età}\PY{l+s+s1}{\PYZsq{}}\PY{p}{:} \PY{n}{np}\PY{o}{.}\PY{n}{random}\PY{o}{.}\PY{n}{randint}\PY{p}{(}\PY{l+m+mi}{18}\PY{p}{,} \PY{l+m+mi}{65}\PY{p}{,} \PY{n}{size}\PY{o}{=}\PY{l+m+mi}{500}\PY{p}{)}\PY{p}{,}
    \PY{l+s+s1}{\PYZsq{}}\PY{l+s+s1}{Soddisfazione}\PY{l+s+s1}{\PYZsq{}}\PY{p}{:} \PY{n}{np}\PY{o}{.}\PY{n}{random}\PY{o}{.}\PY{n}{choice}\PY{p}{(}\PY{p}{[}\PY{l+s+s1}{\PYZsq{}}\PY{l+s+s1}{Molto Soddisfatto}\PY{l+s+s1}{\PYZsq{}}\PY{p}{,} \PY{l+s+s1}{\PYZsq{}}\PY{l+s+s1}{Soddisfatto}\PY{l+s+s1}{\PYZsq{}}\PY{p}{,} \PY{l+s+s1}{\PYZsq{}}\PY{l+s+s1}{Neutro}\PY{l+s+s1}{\PYZsq{}}\PY{p}{,} \PY{l+s+s1}{\PYZsq{}}\PY{l+s+s1}{Insoddisfatto}\PY{l+s+s1}{\PYZsq{}}\PY{p}{,} \PY{l+s+s1}{\PYZsq{}}\PY{l+s+s1}{Molto Insoddisfatto}\PY{l+s+s1}{\PYZsq{}}\PY{p}{]}\PY{p}{,} \PY{n}{size}\PY{o}{=}\PY{l+m+mi}{500}\PY{p}{)}
\PY{p}{\PYZcb{}}

\PY{n}{df} \PY{o}{=} \PY{n}{pd}\PY{o}{.}\PY{n}{DataFrame}\PY{p}{(}\PY{n}{data}\PY{p}{)}
\PY{n+nb}{print}\PY{p}{(}\PY{n}{df}\PY{p}{)}
\PY{n}{conditional\PYZus{}means} \PY{o}{=} \PY{n}{df}\PY{o}{.}\PY{n}{groupby}\PY{p}{(}\PY{l+s+s1}{\PYZsq{}}\PY{l+s+s1}{Soddisfazione}\PY{l+s+s1}{\PYZsq{}}\PY{p}{)}\PY{p}{[}\PY{l+s+s1}{\PYZsq{}}\PY{l+s+s1}{Età}\PY{l+s+s1}{\PYZsq{}}\PY{p}{]}\PY{o}{.}\PY{n}{transform}\PY{p}{(}\PY{l+s+s1}{\PYZsq{}}\PY{l+s+s1}{mean}\PY{l+s+s1}{\PYZsq{}}\PY{p}{)}

\PY{n}{df}\PY{p}{[}\PY{l+s+s1}{\PYZsq{}}\PY{l+s+s1}{Numeric\PYZus{}Var}\PY{l+s+s1}{\PYZsq{}}\PY{p}{]} \PY{o}{=} \PY{n}{conditional\PYZus{}means}
\PY{n+nb}{print}\PY{p}{(}\PY{n}{df}\PY{p}{)}

\PY{c+c1}{\PYZsh{} Crea un grafico a barre per mostrare la media condizionata per ogni categoria}
\PY{n}{plt}\PY{o}{.}\PY{n}{figure}\PY{p}{(}\PY{n}{figsize}\PY{o}{=}\PY{p}{(}\PY{l+m+mi}{8}\PY{p}{,} \PY{l+m+mi}{6}\PY{p}{)}\PY{p}{)}
\PY{n}{sns}\PY{o}{.}\PY{n}{barplot}\PY{p}{(}\PY{n}{data}\PY{o}{=}\PY{n}{df}\PY{p}{,} \PY{n}{x}\PY{o}{=}\PY{l+s+s1}{\PYZsq{}}\PY{l+s+s1}{Soddisfazione}\PY{l+s+s1}{\PYZsq{}}\PY{p}{,} \PY{n}{y}\PY{o}{=}\PY{l+s+s1}{\PYZsq{}}\PY{l+s+s1}{Numeric\PYZus{}Var}\PY{l+s+s1}{\PYZsq{}}\PY{p}{,} \PY{n}{errorbar}\PY{o}{=}\PY{k+kc}{None}\PY{p}{)}
\PY{n}{plt}\PY{o}{.}\PY{n}{xlabel}\PY{p}{(}\PY{l+s+s1}{\PYZsq{}}\PY{l+s+s1}{Soddisfazione}\PY{l+s+s1}{\PYZsq{}}\PY{p}{)}
\PY{n}{plt}\PY{o}{.}\PY{n}{ylabel}\PY{p}{(}\PY{l+s+s1}{\PYZsq{}}\PY{l+s+s1}{Media Condizionata di Numeric\PYZus{}Var}\PY{l+s+s1}{\PYZsq{}}\PY{p}{)}
\PY{n}{plt}\PY{o}{.}\PY{n}{title}\PY{p}{(}\PY{l+s+s1}{\PYZsq{}}\PY{l+s+s1}{Media Condizionata delle Variabili Numeriche per Categoria}\PY{l+s+s1}{\PYZsq{}}\PY{p}{)}
\PY{n}{plt}\PY{o}{.}\PY{n}{xticks}\PY{p}{(}\PY{n}{rotation}\PY{o}{=}\PY{l+m+mi}{90}\PY{p}{)}

\PY{n}{plt}\PY{o}{.}\PY{n}{show}\PY{p}{(}\PY{p}{)}
\end{Verbatim}
\end{tcolorbox}

    \begin{Verbatim}[commandchars=\\\{\}]
     Età        Soddisfazione
0     56    Molto Soddisfatto
1     46  Molto Insoddisfatto
2     32               Neutro
3     60               Neutro
4     25  Molto Insoddisfatto
..   {\ldots}                  {\ldots}
495   37    Molto Soddisfatto
496   41    Molto Soddisfatto
497   29    Molto Soddisfatto
498   52    Molto Soddisfatto
499   50    Molto Soddisfatto

[500 rows x 2 columns]
     Età        Soddisfazione  Numeric\_Var
0     56    Molto Soddisfatto    41.651376
1     46  Molto Insoddisfatto    40.054054
2     32               Neutro    41.747368
3     60               Neutro    41.747368
4     25  Molto Insoddisfatto    40.054054
..   {\ldots}                  {\ldots}          {\ldots}
495   37    Molto Soddisfatto    41.651376
496   41    Molto Soddisfatto    41.651376
497   29    Molto Soddisfatto    41.651376
498   52    Molto Soddisfatto    41.651376
499   50    Molto Soddisfatto    41.651376

[500 rows x 3 columns]
    \end{Verbatim}

    \begin{center}
    \adjustimage{max size={0.9\linewidth}{0.9\paperheight}}{output_91_1.png}
    \end{center}
    { \hspace*{\fill} \\}
    
    \begin{quote}
Il codice fa quanto segue:

Importa le librerie necessarie: pandas, numpy, matplotlib.pyplot e
seaborn.

Imposta un seed per il generatore di numeri casuali di numpy, in modo
che i risultati siano riproducibili.

Genera un dataset di 500 righe con due variabili: `Età' e
`Soddisfazione'.

I dati sono generati utilizzando diverse funzioni: np.random.randint
genera numeri interi casuali e np.random.choice sceglie casualmente tra
le opzioni specificate.

Crea un DataFrame pandas df da questo dataset e lo stampa.

Calcola la media condizionata dell'età per ogni livello di soddisfazione
utilizzando il metodo groupby e transform(`mean'). Questo crea una nuova
serie conditional\_means che ha la stessa lunghezza del DataFrame df.

Aggiunge la serie conditional\_means al DataFrame df come una nuova
colonna `Numeric\_Var' e stampa il DataFrame aggiornato.

Crea una nuova figura con dimensioni specificate (8x6).

Utilizza la funzione barplot della libreria seaborn per creare un
grafico a barre che mostra la media condizionata di `Numeric\_Var' per
ogni livello di `Soddisfazione'.

Imposta le etichette degli assi x e y e il titolo del grafico.

Ruota le etichette dell'asse x di 90 gradi per una migliore leggibilità.

Infine, il comando plt.show() viene utilizzato per visualizzare il
grafico.
\end{quote}

\begin{quote}
Quindi, il codice genera un DataFrame di dati casuali, calcola la media
condizionata dell'età per ogni livello di soddisfazione, e crea e
visualizza un grafico a barre che mostra queste medie condizionate.
\end{quote}

    \subsubsection{Calcolo e Visualizzazione di una Mappa Termica della
Matrice di Correlazione per un DataFrame di Dati
Casuali}\label{calcolo-e-visualizzazione-di-una-mappa-termica-della-matrice-di-correlazione-per-un-dataframe-di-dati-casuali}

    \begin{tcolorbox}[breakable, size=fbox, boxrule=1pt, pad at break*=1mm,colback=cellbackground, colframe=cellborder]
\prompt{In}{incolor}{31}{\boxspacing}
\begin{Verbatim}[commandchars=\\\{\}]
\PY{c+c1}{\PYZsh{}\PYZsh{} import numpy as np}
\PY{k+kn}{import} \PY{n+nn}{seaborn} \PY{k}{as} \PY{n+nn}{sns}
\PY{k+kn}{import} \PY{n+nn}{matplotlib}\PY{n+nn}{.}\PY{n+nn}{pyplot} \PY{k}{as} \PY{n+nn}{plt}

\PY{c+c1}{\PYZsh{} Genera un dataset di esempio con variabili numeriche}
\PY{n}{np}\PY{o}{.}\PY{n}{random}\PY{o}{.}\PY{n}{seed}\PY{p}{(}\PY{l+m+mi}{42}\PY{p}{)}
\PY{n}{data} \PY{o}{=} \PY{n}{pd}\PY{o}{.}\PY{n}{DataFrame}\PY{p}{(}\PY{n}{np}\PY{o}{.}\PY{n}{random}\PY{o}{.}\PY{n}{rand}\PY{p}{(}\PY{l+m+mi}{100}\PY{p}{,} \PY{l+m+mi}{5}\PY{p}{)}\PY{p}{,} \PY{n}{columns}\PY{o}{=}\PY{p}{[}\PY{l+s+s1}{\PYZsq{}}\PY{l+s+s1}{Var1}\PY{l+s+s1}{\PYZsq{}}\PY{p}{,} \PY{l+s+s1}{\PYZsq{}}\PY{l+s+s1}{Var2}\PY{l+s+s1}{\PYZsq{}}\PY{p}{,} \PY{l+s+s1}{\PYZsq{}}\PY{l+s+s1}{Var3}\PY{l+s+s1}{\PYZsq{}}\PY{p}{,} \PY{l+s+s1}{\PYZsq{}}\PY{l+s+s1}{Var4}\PY{l+s+s1}{\PYZsq{}}\PY{p}{,} \PY{l+s+s1}{\PYZsq{}}\PY{l+s+s1}{Var5}\PY{l+s+s1}{\PYZsq{}}\PY{p}{]}\PY{p}{)}

\PY{c+c1}{\PYZsh{} Aggiungi alcune variabili categoriche generate casualmente}
\PY{n}{data}\PY{p}{[}\PY{l+s+s1}{\PYZsq{}}\PY{l+s+s1}{Categoria1}\PY{l+s+s1}{\PYZsq{}}\PY{p}{]} \PY{o}{=} \PY{n}{np}\PY{o}{.}\PY{n}{random}\PY{o}{.}\PY{n}{choice}\PY{p}{(}\PY{p}{[}\PY{l+s+s1}{\PYZsq{}}\PY{l+s+s1}{A}\PY{l+s+s1}{\PYZsq{}}\PY{p}{,} \PY{l+s+s1}{\PYZsq{}}\PY{l+s+s1}{B}\PY{l+s+s1}{\PYZsq{}}\PY{p}{,} \PY{l+s+s1}{\PYZsq{}}\PY{l+s+s1}{C}\PY{l+s+s1}{\PYZsq{}}\PY{p}{]}\PY{p}{,} \PY{n}{size}\PY{o}{=}\PY{l+m+mi}{100}\PY{p}{)}
\PY{n}{data}\PY{p}{[}\PY{l+s+s1}{\PYZsq{}}\PY{l+s+s1}{Categoria2}\PY{l+s+s1}{\PYZsq{}}\PY{p}{]} \PY{o}{=} \PY{n}{np}\PY{o}{.}\PY{n}{random}\PY{o}{.}\PY{n}{choice}\PY{p}{(}\PY{p}{[}\PY{l+s+s1}{\PYZsq{}}\PY{l+s+s1}{X}\PY{l+s+s1}{\PYZsq{}}\PY{p}{,} \PY{l+s+s1}{\PYZsq{}}\PY{l+s+s1}{Y}\PY{l+s+s1}{\PYZsq{}}\PY{p}{]}\PY{p}{,} \PY{n}{size}\PY{o}{=}\PY{l+m+mi}{100}\PY{p}{)}

\PY{c+c1}{\PYZsh{}calcola la matrice di correlazione tra tutte le variabili numerici (correlazione = )}
\PY{n}{correlation\PYZus{}matrix} \PY{o}{=} \PY{n}{data}\PY{o}{.}\PY{n}{corr}\PY{p}{(}\PY{p}{)}

\PY{c+c1}{\PYZsh{}visualizza la matrice di correlazione come heatmap}
\PY{n}{plt}\PY{o}{.}\PY{n}{figure}\PY{p}{(}\PY{n}{figsize}\PY{o}{=}\PY{p}{(}\PY{l+m+mi}{10}\PY{p}{,}\PY{l+m+mi}{8}\PY{p}{)}\PY{p}{)}
\PY{n}{sns}\PY{o}{.}\PY{n}{heatmap}\PY{p}{(}\PY{n}{correlation\PYZus{}matrix}\PY{p}{,} \PY{n}{annot}\PY{o}{=}\PY{k+kc}{True}\PY{p}{,} \PY{n}{cmap}\PY{o}{=}\PY{l+s+s1}{\PYZsq{}}\PY{l+s+s1}{coolwarm}\PY{l+s+s1}{\PYZsq{}}\PY{p}{,} \PY{n}{fmt}\PY{o}{=}\PY{l+s+s2}{\PYZdq{}}\PY{l+s+s2}{.2f}\PY{l+s+s2}{\PYZdq{}}\PY{p}{,} \PY{n}{alpha}\PY{o}{=}\PY{l+m+mf}{0.7}\PY{p}{)}

\PY{n}{plt}\PY{o}{.}\PY{n}{title}\PY{p}{(}\PY{l+s+s1}{\PYZsq{}}\PY{l+s+s1}{Matrice di correlazione}\PY{l+s+s1}{\PYZsq{}}\PY{p}{)}
\PY{n}{plt}\PY{o}{.}\PY{n}{show}\PY{p}{(}\PY{p}{)}
\end{Verbatim}
\end{tcolorbox}

    \begin{Verbatim}[commandchars=\\\{\}]
C:\textbackslash{}Users\textbackslash{}joele\textbackslash{}AppData\textbackslash{}Local\textbackslash{}Temp\textbackslash{}ipykernel\_21328\textbackslash{}1612240019.py:14:
FutureWarning:

The default value of numeric\_only in DataFrame.corr is deprecated. In a future
version, it will default to False. Select only valid columns or specify the
value of numeric\_only to silence this warning.

    \end{Verbatim}

    \begin{center}
    \adjustimage{max size={0.9\linewidth}{0.9\paperheight}}{output_94_1.png}
    \end{center}
    { \hspace*{\fill} \\}
    
    \begin{quote}
Il codice fa quanto segue:

Importa le librerie necessarie: numpy, seaborn, matplotlib.pyplot e
pandas.

Imposta un seed per il generatore di numeri casuali di numpy, in modo
che i risultati siano riproducibili.

Genera un DataFrame data di 100 righe con cinque variabili numeriche:
`Var1', `Var2', `Var3', `Var4' e `Var5'. I dati sono generati
utilizzando una distribuzione uniforme continua tra 0 e 1.

Aggiunge due variabili categoriche al DataFrame: `Categoria1' e
`Categoria2'. I dati sono generati scegliendo casualmente tra le opzioni
specificate.

Calcola la matrice di correlazione tra tutte le variabili numeriche
utilizzando il metodo corr(). Questo restituisce un nuovo DataFrame
correlation\_matrix in cui l'indice e le colonne sono i nomi delle
variabili numeriche e ogni cella contiene il coefficiente di
correlazione tra due variabili.

Crea una nuova figura con dimensioni specificate (10x8).

Utilizza la funzione heatmap della libreria seaborn per creare una mappa
termica (heatmap) della matrice di correlazione. La mappa termica è
colorata usando il colormap `coolwarm', e ogni cella è annotata con il
coefficiente di correlazione corrispondente, formattato come un numero
decimale con due cifre decimali.

Imposta il titolo della mappa termica come `Matrice di correlazione'.

Infine, il comando plt.show() viene utilizzato per visualizzare la
figura.
\end{quote}

\begin{quote}
Quindi, il codice genera un DataFrame di dati casuali, calcola la
matrice di correlazione tra le variabili numeriche e crea e visualizza
una mappa termica di questa matrice di correlazione.
\end{quote}

    \subsubsection{Identificazione e Conteggio delle Righe con Dati Mancanti
nel
DataFrame}\label{identificazione-e-conteggio-delle-righe-con-dati-mancanti-nel-dataframe}

    \begin{tcolorbox}[breakable, size=fbox, boxrule=1pt, pad at break*=1mm,colback=cellbackground, colframe=cellborder]
\prompt{In}{incolor}{32}{\boxspacing}
\begin{Verbatim}[commandchars=\\\{\}]
\PY{c+c1}{\PYZsh{}identificazione delle righe con dati mancati}
\PY{n}{righe\PYZus{}con\PYZus{}dati\PYZus{}mancanti} \PY{o}{=} \PY{n}{df}\PY{p}{[}\PY{n}{df}\PY{o}{.}\PY{n}{isnull}\PY{p}{(}\PY{p}{)}\PY{o}{.}\PY{n}{any}\PY{p}{(}\PY{n}{axis}\PY{o}{=}\PY{l+m+mi}{1}\PY{p}{)}\PY{p}{]}
\PY{n+nb}{len}\PY{p}{(}\PY{n}{righe\PYZus{}con\PYZus{}dati\PYZus{}mancanti}\PY{p}{)}
\end{Verbatim}
\end{tcolorbox}

            \begin{tcolorbox}[breakable, size=fbox, boxrule=.5pt, pad at break*=1mm, opacityfill=0]
\prompt{Out}{outcolor}{32}{\boxspacing}
\begin{Verbatim}[commandchars=\\\{\}]
0
\end{Verbatim}
\end{tcolorbox}
        
    \begin{quote}
Il codice fa quanto segue:

Utilizza il metodo isnull().any(axis=1) sul DataFrame df per creare una
serie booleana che indica se c'è un valore mancante (NaN) in qualsiasi
colonna di ciascuna riga.

Seleziona le righe del DataFrame df che hanno almeno un valore mancante,
utilizzando la serie booleana come indice. Questo restituisce un nuovo
DataFrame righe\_con\_dati\_mancanti che contiene solo le righe con
valori mancanti.

Infine, calcola e stampa il numero di righe nel DataFrame
righe\_con\_dati\_mancanti utilizzando la funzione len().
\end{quote}

\begin{quote}
Quindi, il codice identifica le righe con dati mancanti nel DataFrame df
e stampa il loro numero.
\end{quote}

    \subsubsection{Percentuale di valori mancanti per
colonna}\label{percentuale-di-valori-mancanti-per-colonna}

    \begin{tcolorbox}[breakable, size=fbox, boxrule=1pt, pad at break*=1mm,colback=cellbackground, colframe=cellborder]
\prompt{In}{incolor}{33}{\boxspacing}
\begin{Verbatim}[commandchars=\\\{\}]
\PY{n}{missing\PYZus{}percent} \PY{o}{=} \PY{p}{(}\PY{n}{df}\PY{o}{.}\PY{n}{isnull}\PY{p}{(}\PY{p}{)}\PY{o}{.}\PY{n}{sum}\PY{p}{(}\PY{p}{)} \PY{o}{/} \PY{n+nb}{len}\PY{p}{(}\PY{n}{df}\PY{p}{)} \PY{o}{*} \PY{l+m+mi}{100}\PY{p}{)}
\PY{n}{missing\PYZus{}percent}
\end{Verbatim}
\end{tcolorbox}

            \begin{tcolorbox}[breakable, size=fbox, boxrule=.5pt, pad at break*=1mm, opacityfill=0]
\prompt{Out}{outcolor}{33}{\boxspacing}
\begin{Verbatim}[commandchars=\\\{\}]
Età              0.0
Soddisfazione    0.0
Numeric\_Var      0.0
dtype: float64
\end{Verbatim}
\end{tcolorbox}
        
    \begin{quote}
Questo codice calcola la percentuale di valori mancanti in ciascuna
colonna del DataFrame df:

df.isnull() restituisce un DataFrame dello stesso formato di df, ma con
valori True dove i valori originali erano mancanti (NaN) e False dove i
valori erano presenti.

.sum() somma i valori True (considerati come 1) in ciascuna colonna,
restituendo il numero totale di valori mancanti per colonna.

Dividendo per len(df) si ottiene la proporzione di valori mancanti per
colonna, poiché len(df) restituisce il numero totale di righe nel
DataFrame.

Moltiplicando per 100, si converte la proporzione in percentuale.
\end{quote}

\begin{quote}
Il risultato, missing\_percent, è una serie pandas in cui l'indice è
l'elenco delle colonne di df e i valori sono le percentuali di valori
mancanti per ciascuna colonna.
\end{quote}

    \subsubsection{Visualizzazione dei Valori Mancanti con
Heatmap}\label{visualizzazione-dei-valori-mancanti-con-heatmap}

    \begin{tcolorbox}[breakable, size=fbox, boxrule=1pt, pad at break*=1mm,colback=cellbackground, colframe=cellborder]
\prompt{In}{incolor}{34}{\boxspacing}
\begin{Verbatim}[commandchars=\\\{\}]
\PY{n}{missing\PYZus{}matrix} \PY{o}{=} \PY{n}{df}\PY{o}{.}\PY{n}{isnull}\PY{p}{(}\PY{p}{)}
\PY{c+c1}{\PYZsh{}crea una heatmap colorata}
\PY{n}{plt}\PY{o}{.}\PY{n}{figure}\PY{p}{(}\PY{n}{figsize}\PY{o}{=}\PY{p}{(}\PY{l+m+mi}{8}\PY{p}{,}\PY{l+m+mi}{6}\PY{p}{)}\PY{p}{)}
\PY{n}{sns}\PY{o}{.}\PY{n}{heatmap}\PY{p}{(}\PY{n}{missing\PYZus{}matrix}\PY{p}{,} \PY{n}{cmap}\PY{o}{=}\PY{l+s+s1}{\PYZsq{}}\PY{l+s+s1}{viridis}\PY{l+s+s1}{\PYZsq{}}\PY{p}{,} \PY{n}{cbar}\PY{o}{=}\PY{k+kc}{False}\PY{p}{,}\PY{n}{alpha}\PY{o}{=}\PY{l+m+mf}{0.8}\PY{p}{)}
\PY{n}{plt}\PY{o}{.}\PY{n}{title}\PY{p}{(}\PY{l+s+s1}{\PYZsq{}}\PY{l+s+s1}{Matrice di missing values}\PY{l+s+s1}{\PYZsq{}}\PY{p}{)}
\PY{n}{plt}\PY{o}{.}\PY{n}{show}\PY{p}{(}\PY{p}{)}
\end{Verbatim}
\end{tcolorbox}

    \begin{center}
    \adjustimage{max size={0.9\linewidth}{0.9\paperheight}}{output_103_0.png}
    \end{center}
    { \hspace*{\fill} \\}
    
    \begin{quote}
Il codice fa quanto segue:

Crea una matrice di valori mancanti: Il comando missing\_matrix =
df.isnull() crea una nuova matrice chiamata missing\_matrix. Questa
matrice ha la stessa forma del DataFrame originale df, ma contiene
valori True dove i valori originali erano mancanti (NaN) e False dove i
valori erano presenti.
\end{quote}

Crea una heatmap dei valori mancanti: Il comando
plt.figure(figsize=(8,6)) crea una nuova figura con dimensioni
specificate (8x6). Il comando sns.heatmap(missing\_matrix,
cmap=`viridis', cbar=False,alpha=0.8) utilizza la funzione heatmap della
libreria seaborn (alias sns) per creare una heatmap dei valori mancanti.
L'asse x rappresenta le colonne del DataFrame e l'asse y rappresenta le
righe. I valori mancanti sono rappresentati con un colore e i valori non
mancanti con un altro colore. Il comando plt.title(`Matrice di missing
values') imposta il titolo della heatmap come `Matrice di missing
values'. Infine, il comando plt.show() viene utilizzato per visualizzare
la heatmap.

\begin{quote}
Quindi, il codice crea e visualizza una heatmap dei valori mancanti nel
DataFrame df. Questo può essere molto utile per capire dove si
concentrano i dati mancanti.
\end{quote}

    \subsection{IMPORT DATA}\label{import-data}

    \subsubsection{Caricamento e Visualizzazione dei Dati da un Foglio di
Lavoro Excel in
Python}\label{caricamento-e-visualizzazione-dei-dati-da-un-foglio-di-lavoro-excel-in-python}

    \begin{tcolorbox}[breakable, size=fbox, boxrule=1pt, pad at break*=1mm,colback=cellbackground, colframe=cellborder]
\prompt{In}{incolor}{35}{\boxspacing}
\begin{Verbatim}[commandchars=\\\{\}]
\PY{k+kn}{import} \PY{n+nn}{pandas} \PY{k}{as} \PY{n+nn}{pd}

\PY{n}{percorso\PYZus{}file\PYZus{}excel} \PY{o}{=} \PY{l+s+s2}{\PYZdq{}}\PY{l+s+s2}{C:}\PY{l+s+se}{\PYZbs{}\PYZbs{}}\PY{l+s+s2}{Users}\PY{l+s+se}{\PYZbs{}\PYZbs{}}\PY{l+s+s2}{joele}\PY{l+s+se}{\PYZbs{}\PYZbs{}}\PY{l+s+s2}{OneDrive}\PY{l+s+se}{\PYZbs{}\PYZbs{}}\PY{l+s+s2}{Desktop}\PY{l+s+se}{\PYZbs{}\PYZbs{}}\PY{l+s+s2}{dati robotica}\PY{l+s+se}{\PYZbs{}\PYZbs{}}\PY{l+s+s2}{serieA.xlsx}\PY{l+s+s2}{\PYZdq{}}
\PY{n}{df} \PY{o}{=} \PY{n}{pd}\PY{o}{.}\PY{n}{read\PYZus{}excel}\PY{p}{(}\PY{n}{percorso\PYZus{}file\PYZus{}excel} \PY{p}{,} \PY{n}{sheet\PYZus{}name}\PY{o}{=}\PY{l+s+s1}{\PYZsq{}}\PY{l+s+s1}{10\PYZhy{}11}\PY{l+s+s1}{\PYZsq{}}\PY{p}{)}
\PY{n}{df}
\end{Verbatim}
\end{tcolorbox}

            \begin{tcolorbox}[breakable, size=fbox, boxrule=.5pt, pad at break*=1mm, opacityfill=0]
\prompt{Out}{outcolor}{35}{\boxspacing}
\begin{Verbatim}[commandchars=\\\{\}]
    position                   team  Pt  Played  Won  Net  lose  Goals made  \textbackslash{}
0          1            Milan Milan  82      38   24   10     4          65
1          2            Inter Inter  76      38   23    7     8          69
2          3          Napoli Napoli  70      38   21    7    10          59
3          4        Udinese Udinese  66      38   20    6    12          65
4          5            Lazio Lazio  66      38   20    6    12          55
5          6              Roma Roma  63      38   18    9    11          59
6          7      Juventus Juventus  58      38   15   13    10          57
7          8        Palermo Palermo  56      38   17    5    16          58
8          9  Fiorentina Fiorentina  51      38   12   15    11          49
9         10            Genoa Genoa  51      38   14    9    15          45
10        11          Chievo Chievo  46      38   11   13    14          38
11        12            Parma Parma  46      38   11   13    14          39
12        13        Catania Catania  46      38   12   10    16          40
13        14      Cagliari Cagliari  45      38   12    9    17          44
14        15          Cesena Cesena  43      38   11   10    17          38
15        16   Bologna Bologna (-3)  42      38   11   12    15          35
16        17            Lecce Lecce  41      38   11    8    19          46
17        18    Sampdoria Sampdoria  36      38    8   12    18          33
18        19        Brescia Brescia  32      38    7   11    20          34
19        20              Bari Bari  24      38    5    9    24          27

    Goals suffered  Difference goals
0               24                41
1               42                27
2               39                20
3               43                22
4               39                16
5               52                 7
6               47                10
7               63                -5
8               44                 5
9               47                -2
10              40                -2
11              47                -8
12              52               -12
13              51                -7
14              50               -12
15              52               -17
16              66               -20
17              49               -16
18              52               -18
19              56               -29
\end{Verbatim}
\end{tcolorbox}
        
    \begin{quote}
il codice sta facendo quanto segue:

Importa la libreria pandas: La prima riga del codice importa la libreria
pandas e la rinomina come pd. Pandas è una libreria di Python che
fornisce strutture dati e funzionalità per manipolare e analizzare i
dati.

Specifica il percorso del file Excel: La variabile percorso\_file\_excel
viene impostata con il percorso del file Excel che desideri leggere. In
questo caso, il file si trova nella cartella `dati robotica' sul tuo
desktop.

Legge il file Excel: La funzione pd.read\_excel() viene utilizzata per
leggere il file Excel specificato. Il parametro sheet\_name=`10-11'
indica che desideri leggere il foglio di lavoro chiamato `10-11' nel
file Excel.

Visualizza il DataFrame: L'ultima riga del codice, df, stampa il
contenuto del DataFrame df. Un DataFrame è una struttura dati
bidimensionale, simile a una tabella, che può contenere dati di vari
tipi (numerici, stringhe, booleani, ecc.) e consente operazioni di
manipolazione dei dati come quelle in SQL e Excel.
\end{quote}

\begin{quote}
In sintesi, il codice legge un foglio di lavoro specifico da un file
Excel e lo carica in un DataFrame di pandas, che viene poi visualizzato.
\end{quote}

    \subsubsection{Caricamento e Visualizzazione dei Dati da un File CSV in
Python}\label{caricamento-e-visualizzazione-dei-dati-da-un-file-csv-in-python}

    \begin{tcolorbox}[breakable, size=fbox, boxrule=1pt, pad at break*=1mm,colback=cellbackground, colframe=cellborder]
\prompt{In}{incolor}{36}{\boxspacing}
\begin{Verbatim}[commandchars=\\\{\}]
\PY{k+kn}{import} \PY{n+nn}{pandas} \PY{k}{as} \PY{n+nn}{pd}
\PY{k+kn}{import} \PY{n+nn}{numpy} \PY{k}{as} \PY{n+nn}{np}
\PY{k+kn}{import} \PY{n+nn}{matplotlib}\PY{n+nn}{.}\PY{n+nn}{pyplot} \PY{k}{as} \PY{n+nn}{plt}
\PY{k+kn}{import} \PY{n+nn}{seaborn} \PY{k}{as} \PY{n+nn}{sns}
\PY{n}{percorso\PYZus{}file\PYZus{}csv} \PY{o}{=} \PY{l+s+s2}{\PYZdq{}}\PY{l+s+s2}{C:}\PY{l+s+se}{\PYZbs{}\PYZbs{}}\PY{l+s+s2}{Users}\PY{l+s+se}{\PYZbs{}\PYZbs{}}\PY{l+s+s2}{joele}\PY{l+s+se}{\PYZbs{}\PYZbs{}}\PY{l+s+s2}{OneDrive}\PY{l+s+se}{\PYZbs{}\PYZbs{}}\PY{l+s+s2}{Desktop}\PY{l+s+se}{\PYZbs{}\PYZbs{}}\PY{l+s+s2}{dati robotica}\PY{l+s+se}{\PYZbs{}\PYZbs{}}\PY{l+s+s2}{pokemons.csv}\PY{l+s+s2}{\PYZdq{}}
\PY{n}{df} \PY{o}{=} \PY{n}{pd}\PY{o}{.}\PY{n}{read\PYZus{}csv}\PY{p}{(}\PY{n}{percorso\PYZus{}file\PYZus{}csv}\PY{p}{)}
\PY{n+nb}{print}\PY{p}{(}\PY{n}{df}\PY{o}{.}\PY{n}{head}\PY{p}{(}\PY{p}{)}\PY{p}{)}
\end{Verbatim}
\end{tcolorbox}

    \begin{Verbatim}[commandchars=\\\{\}]
   id        name      rank    generation evolves\_from  type1   type2  hp  \textbackslash{}
0   1   bulbasaur  ordinary  generation-i      nothing  grass  poison  45
1   2     ivysaur  ordinary  generation-i    bulbasaur  grass  poison  60
2   3    venusaur  ordinary  generation-i      ivysaur  grass  poison  80
3   4  charmander  ordinary  generation-i      nothing   fire    None  39
4   5  charmeleon  ordinary  generation-i   charmander   fire    None  58

   atk  def  spatk  spdef  speed  total  height  weight  \textbackslash{}
0   49   49     65     65     45    318       7      69
1   62   63     80     80     60    405      10     130
2   82   83    100    100     80    525      20    1000
3   52   43     60     50     65    309       6      85
4   64   58     80     65     80    405      11     190

               abilities                                               desc
0  overgrow chlorophyll   A strange seed was planted on its back at birt{\ldots}
1  overgrow chlorophyll   When the bulb on its back grows large, it appe{\ldots}
2  overgrow chlorophyll   The plant blooms when it is absorbing solar en{\ldots}
3     blaze solar-power   Obviously prefers hot places. When it rains, s{\ldots}
4     blaze solar-power   When it swings its burning tail, it elevates t{\ldots}
    \end{Verbatim}

    \begin{quote}
il codice fa quanto segue:

Importa le librerie necessarie: Le prime righe del codice importano le
librerie necessarie: pandas (rinominata come pd), numpy (rinominata come
np), matplotlib.pyplot (rinominata come plt) e seaborn (rinominata come
sns). Queste librerie forniscono funzionalità per manipolare e
analizzare i dati, così come per la visualizzazione dei dati.

Specifica il percorso del file CSV: La variabile percorso\_file\_csv
viene impostata con il percorso del file CSV che desideri leggere. In
questo caso, il file si trova nella cartella `dati robotica' sul tuo
desktop.

Legge il file CSV: La funzione pd.read\_csv() viene utilizzata per
leggere il file CSV specificato. Questa funzione legge il file CSV e lo
converte in un DataFrame di pandas.

Visualizza il DataFrame: L'ultima riga del codice, print(df.head()),
stampa le prime 5 righe del DataFrame df. Un DataFrame è una struttura
dati bidimensionale, simile a una tabella, che può contenere dati di
vari tipi (numerici, stringhe, booleani, ecc.) e consente operazioni di
manipolazione dei dati.
\end{quote}

\begin{quote}
In sintesi, il codice legge un file CSV specifico e lo carica in un
DataFrame di pandas, che viene poi visualizzato stampando le prime 5
righe.
\end{quote}


    % Add a bibliography block to the postdoc
    
    
    
\end{document}
